%Texlive-full Version 3.141592-1.40.3 (Web2C 7.5.6)
%Kile Version 2.0.83

\documentclass[a4paper,10pt]{article}
\usepackage[utf8x]{inputenc}

\usepackage{lmodern}
\usepackage[a4paper]{geometry}

\usepackage{hyperref}

\usepackage{amsmath}
\usepackage{amssymb}

\usepackage{pstricks}
\usepackage{pst-node}
\usepackage{pst-plot}
\usepackage{enumerate}
\usepackage{textcomp}

\begin{document}
%%%%%%%%%%%%%%%%%% DOCUMENT TITLE %%%%%%%%%%%%%%%%%%%%%%%%%
\begin{center}Memento Basis Requirement knowledge for Laure Elie Master\end{center}
%%%%%%%%%%%%%%%%%% DOCUMENT TITLE %%%%%%%%%%%%%%%%%%%%%%%%%
\section{Topology Notions}
\paragraph{Topological separated space} called also Hausdorff space or T$_2$-space is a topological space in which distinct points have disjoint neighbourhoods. All metric space are separated.

\paragraph{Countable set} $K$ is a countable set if either it is empty or there exists a surjection from $\mathbb{N}$ onto $K$ 

\paragraph{Complete metric space} A metric space $\textbf{X}$ is complete if every Cauchy sequence in $\textbf{X}$ converges in $\textbf{X}$ 

\paragraph{Topological separable space} A topological space having a countable dense subset.

\paragraph{Topological countable basis} A subset of the topology that an arbitrary open sets can be written as countable union of this subset

\paragraph{Polish space} is a separable complete metric space.

\subsection{Banach space}
\paragraph{Definition} A Banach space is a complete normed vector space, noting ($\mathcal{B}, \|.\|$). Banach space become a Hilbert space if and only if the norm satisfy the parallelogram identity
\[
\| a+b \|^2  + \| a-b \|^2 = 2\|a\|^2  + 2\|b\|^2 
\]

\subsection{Hilbert space} 
\paragraph{Definition} Hilbert space is a Banach space having a inner product $<.,.>$ associating to its norm
\[
\| x \| = \sqrt{<x,x>}
\]

\paragraph{Sublinear function} Let $V$ a vector space, $p:V\longrightarrow \mathbb{R}$. Then $p$ is a sublinear function iff
\begin{itemize}
 \item $p$ is positive homogene : $p(\alpha v) = \alpha p(v) \hspace{3mm} \forall \alpha \geq 0$
 \item $p$ is subadditive : $p(u+v) \leq p(u)+ p(v) \hspace{3mm} \forall u,v \in V$ 
\end{itemize}

\subsection{Compactness}
\paragraph{Definition} A topological space $E$ is compact if it is separated and verify the Borel-Lebesgue proprety expressed equivalently in three ways, i.e 
\begin{itemize}
\renewcommand{\labelitemi}{$\vcenter{\hbox{\tiny$\bullet$}}$}
 \item Each open covers of $E$ has finite subcover.
 \item Each familly of closed sets with empty intersection, we can extrait a finite subfamilly with also empty intersection.
 \item Each familly of closed sets in which every finite subfamilly are of non-empty intersection, is itself of non-empty intersection.
\end{itemize}
A subset $K$ of a topological space $E$ is called compact iff it is compact in the induced topology.
\paragraph{Propreties}
\begin{enumerate}
 \item All finite space are compact.
 \item Every sequence of points in a compact space has an accumulation point.
 \item If a sequence in a compact space has a unique accumulation point, it converges.
 \item Each closed subset in a compact space is compact.
 \item Each compact subset in a separated space is closed.
 \item In a separated space, every finite union or intersection of compacts are compact.
 \item An image of a continuous function from a compact to a separated space is a compact.
 \item A continuous real-valued function on a non-empty compact space is bounded and attains its supremum.
 \item \textbf{Tychonoff theorem : } The product of any collection of compact space is compact.
 \item In $\mathbb{R}^n$ a subset is compact iff it is closed and bouned.   
\end{enumerate}

\paragraph{Locally compact space} A topological space is locally compact if it is separated and each point has a compact neighbourhood.
\begin{itemize}
 \item Every compact space is locally compact.
 \item $\mathbb{R}^n$ is locally compact.
 \item In a locally compact space, every open, closed subset are locally compact.
 \item In a separated space, intersection of two locally compact subset is locally compact.\textit{ It is not the case for union.}
 \item Finite product of locally compact space is locally compact.
\end{itemize}

\paragraph{Sequentially compact space} A topological is sequentially compact if every infinite sequence has a convergent subsequence. In metric spaces, there are equivalent of \textit{sequentially compact} and \textit{compact}.  

\section{Mesurable space}
\paragraph{$\sigma$-algebra} Giving a set $\textbf{X}$, a $\sigma$-algebra of $\textbf{X}$ is a familly of subsets $\Sigma$ satisfying 
\begin{enumerate}
 \item $\emptyset \in \Sigma$
 \item $\Sigma$ is closed under complement : $ E \in \Sigma \Longrightarrow E^{c} \in \Sigma$
 \item $\Sigma$ is closed under countable union : $ \langle E_{n} \rangle_{n \in \mathbb{N}} \in \Sigma \Longrightarrow \cup E_{n} \in \Sigma $ 
\end{enumerate}
It imply that $\sigma$-algebra is stable by other set operators as difference, symetric difference, finite intersection, union, $\textit{countable}$ intersection, union. 

\paragraph{$\lambda$-systems (Dynkin system)} Giving a set $\textbf{X}$, a $\lambda$-system of $\textbf{X}$ is a familly of subsets $\varLambda$ satisfying 
\begin{enumerate}
 \item $\textbf{X} \in \varLambda$
 \item $\varLambda$ is closed under relative complement : $ A,B \in \varLambda, A \subset B \Longrightarrow B \backslash A \in \varLambda$
 \item $\varLambda$ is closed under increasing countable union : $ \langle A_{n} \rangle_{n \in \mathbb{N}} \in \varLambda, A_n \subset A_{n+1} \Longrightarrow \cup A_{n} \in \varLambda $ 
\end{enumerate}


\paragraph{$\pi$-systems} Giving a set $\textbf{X}$, a $\pi$-system of $\textbf{X}$ is a familly of subsets $\xi$ satisfying 
\begin{enumerate}
 \item $\xi$ is non empty.
 \item $\xi$ is closed under finite intersection.
\end{enumerate}

\paragraph{Ring} Giving a set $\textbf{X}$, a ring of $\textbf{X}$ is a familly of subsets $\textbf{R}$ satisfying 
\begin{enumerate}
 \item $\emptyset \in \textbf{R}$
 \item $\textbf{R}$ is closed under finite union :$ A,B \in \textbf{R} \Longrightarrow A \cup B \in \textbf{R} $
 \item $\textbf{R}$ is closed under relative complement : $ A,B \in \textbf{R} \Longrightarrow B \backslash A \in \textbf{R}$
\end{enumerate}

\paragraph{Transport lemma (generated $\sigma$-algebra by function)} 
\[
\text{Let} \hspace{5mm} 
\left\{ 
\begin{array}{l}
 f : X \longrightarrow Y \\
 \xi \in \mathcal{P}(Y)
\end{array}\right.
\hspace{5mm} \text{then} \hspace{5mm}
\sigma(f^{-1}(\xi)) = f^{-1}(\sigma(\xi))
\]

\subsection{Mesurable function}

\paragraph{Definition}
Let $(X,\mathcal{A})$ and $(Y,\mathcal{B})$ be measurable spaces. A function $f:X \longrightarrow Y$ is said to be measurable iff 
\[
\forall \text{B} \in \mathcal{B} \hbox{ , } f^{-1}(B) \in \mathcal{A} 
\]

\paragraph{Theorem} Let $(X,\mathcal{A})$ and $(Y,\mathcal{B})$ be measurable spaces and $\mathcal{B} = \sigma(\xi)$, then 
\[
f \text{ is measurable } \Longleftrightarrow f^{-1}(\xi) \subset \mathcal{A}
\]
For real functions (in $\mathbb{R}$), every operation like $+,-,*,/,\text{sup},\text{inf},\overline{\text{lim}}, \underline{\text{lim}}$ ... give a measurable function.

\subsection{Mesure Space}
\paragraph{Definition} A Mesure space is a triple $(X,\Sigma,\mu)$ where
\begin{itemize}
 \item $X$ is a set, and $\Sigma$ is the $\sigma$-algebra of $X$
 \item $\mu : \Sigma \longrightarrow \bar{\mathbb{R}}_{+}$ is a positive function such that
 \begin{enumerate}
  \item $\mu(\emptyset) = 0$
  \item if $\langle E_{n} \rangle_{n \in \mathbb{N}}$ is a disjoint sequence set in $\Sigma$, then $\mu(\uplus E_{n}) = \sum_{n \in \mathbb{N}}\mu(E_n) $
 \end{enumerate}
\end{itemize}
Appart general cases, there are several special mesures as point-supported mesure, counting mesure, dirac mesure...

\paragraph{Elementary properties of mesure}
\begin{enumerate}
 \item monotonicity : for all $E,F \in \Sigma$ 
 \begin{itemize}
  \item $E \subseteq F \Longrightarrow \mu(E) \leq \mu(F)$
  \item if $\mu(F) < \infty  \Longrightarrow \mu(E\backslash F) = \mu(E) - \mu(F)$
 \end{itemize}

 \item strong additivity : $\mu(E\cup F) + \mu(E\cap F) = \mu(E) + \mu(F)$ for all $E,F \in \Sigma$ 
 \item left-continuity : if $\langle E_{n} \rangle_{n \in \mathbb{N}}$ a non-decreasing sequence in $\Sigma$ then 
 $\mu(\cup^{\uparrow} E_{n}) = \text{lim}^{\uparrow} \mu(E_{n})$
 \item right-continuity : if $\langle E_{n} \rangle_{n \in \mathbb{N}}$ a non-increasing sequence in $\Sigma$ then 
 $\mu(\cup^{\downarrow} E_{n}) = \text{lim}^{\downarrow} \mu(E_{n})$
 \item subadditivity : for all $\langle E_{n} \rangle_{n \in \mathbb{N}}$ a sequence in $\Sigma$ then $\mu(\cup E_{n}) \leq \sum_{n \in \mathbb{N}}\mu(E_n) $
\end{enumerate}

\paragraph{Negligible sets} $A \subset X$ is a negligible set if $\exists E \in \Sigma, A \subset E, \text{ and } \mu(E)=0$
\paragraph{Conegligible sets} $A \subset X$ is a conegligible set if $X \backslash A$ is negligible.

\paragraph{}There are two fundamental notions in measure theory : the existance and uniqueness of the mesure. 
\begin{itemize}
 \item The uniqueness part is easier, based on the monotone class theorem, state that if two $\sigma$-finite measures coincide on a certain $\pi$-system generating the $\sigma$-field, then these two measures are equal.  
 \item The existance part is build by the outer measure. The idea is by building a pre-measure on a ring, then one can extend to a measure on the $\sigma$-algebra generated by the ring. 
\end{itemize}

\paragraph{Regular measure}
Let $(X, \mathcal{O}(X))$ a topological space, and $\mu$ a measure defined on the Borel algebra $\mathcal{B}(X) = \sigma(\mathcal{O}(X))$. If for all set $A \in \mathcal{B}(X)$ 
\[
\left\|
\begin{array}{l}
 \mu(A) = \text{inf}\{\mu(O) | A \subset O, O \text{ open set}\} \text{ then } \mu \text{ is outer regular } \\
 \mu(A) = \text{sup}\{\mu(K) | K \subset A, K \text{ compact set}\} \text{ then } \mu \text{ is inner regular } 
\end{array}
\right.
\]
A theorem state that all is measure (finite on compacts) on a metric space, locally compact and separable is regular. Especially the Lebesgue measure on $\mathbb{R}^d$, or all measure defined by a density $\mu=f.\lambda_d$ where $f$ is locally integrable (i.e $\int_K fd\lambda_d < \infty \hbox{ , } K \text{compact}$) are all regular.

\paragraph{Outer measure}
An outer measure on $E$ is a function $\mu^{*} : \mathcal{P}(E) \longrightarrow [0,\infty]$ satisfying :
\begin{enumerate}
 \item null emty set : $\mu^{*}(\emptyset) = 0$
 \item monotonicity : for all $A \subset B \Longrightarrow \mu^{*}(A) \leq \mu^{*}(B)$  
 \item countable subadditivity : $\langle A_{n} \rangle_{n \in \mathbb{N}}$ subsets of $X$ then $\mu^{*}(\cup A_{n}) \leq \sum_{n \in \mathbb{N}}\mu^{*}(A_n) $
\end{enumerate}


\subsection{Product measure}
\paragraph{Product $\sigma$-algebra} Let $(X,\mathcal{A})$ and $(Y,\mathcal{B})$ be two measurable spaces. Then the product measurable space is generated by the cartesian product of $\sigma$-algebra, i.e :
\[
\mathcal{A} \otimes \mathcal{B} := \sigma (\{A \times B \hbox{ , } A \in \mathcal{A} \hbox{ , }  B \in \mathcal{B} \})
\]
\paragraph{Propreties of product measurable spaces}
\begin{itemize}
\renewcommand{\labelitemi}{$\vcenter{\hbox{\tiny$\bullet$}}$}
 \item The product measurable space $(X\times Y, \mathcal{A}\otimes\mathcal{B})$ is the smallest $\sigma$-algebra making measurable the canonical projections $\pi_{X}$ and $\pi_{Y}$
 \item The product $\sigma$-algebra is associative, i.e 
  \[
    \mathcal{A} \otimes \mathcal{B} \otimes \mathcal{C} = (\mathcal{A} \otimes \mathcal{B}) \otimes \mathcal{C} = \mathcal{A} \otimes( \mathcal{B} \otimes \mathcal{C})
  \]
  \item In case of Borelian $\sigma$-algebra,
  \begin{enumerate}[i]
   \item $\mathcal{B}(X) \otimes \mathcal{B}(Y) \subset \mathcal{B}(X\times Y)$
   \item if $X$ and $Y$ has a countable base, then $\mathcal{B}(X) \otimes \mathcal{B}(Y) = \mathcal{B}(X\times Y)$
  \end{enumerate}
  \item Givent the product measurable space $(X\times Y, \mathcal{A}\otimes\mathcal{B})$, then for every $C \in \mathcal{A}\otimes\mathcal{B})$, the $x$-section $C_x=\{y\in Y | (x,y) \in C \} \subset Y$  and $y$-section $C_y=\{x\in X | (x,y) \in C \} \subset X$ are respectively in $\mathcal{B}$ and $\mathcal{A}$
\end{itemize}
\paragraph{Measure product} Let $(X,\mathcal{A},\mu)$ and $(Y,\mathcal{B},\nu)$ be two $\sigma$-\textit{finite} measure spaces. Then
\begin{enumerate}
 \item There exists an unique measure $m$ on $(X\times Y, \mathcal{A}\otimes \mathcal{B} )$ verify
 \[
 \forall A\in \mathcal{A} \hbox{ , } B\in\mathcal{B} \hbox{ , } \hspace{5mm} m(A\times B) = \mu(A)\nu(B)
 \]
 It is also a $\sigma$-finite measure noting $\mu \otimes \nu$
 \item $\forall C \in  \mathcal{A}\otimes \mathcal{B} \hspace{5mm}  m(C) = \int_X \nu(C_x)d\mu_x = \int_Y \mu(C_y)d\nu_y $
\end{enumerate}


\section{Integration}
\paragraph{Simple function} is a finite linear combination of indicator functions on a measurable sets. 
\[
\varphi(\omega) = \sum_{k=1}^{n} a_k \mathbf{1}_{A_k}(\omega)
\]
\begin{itemize}
 \item Any non-negative mesurable function is the pointwise limit of a monotonic increasing sequence of non-negative simple functions. 
 \item Any bounded mesurable function is the uniform pointwise limit of a sequence of simple functions.
\end{itemize}
\paragraph{Lebesgue integration } for a non-negative measurable function $f$ :
\[
\int fd\mu = \text{sup} \{ \int \varphi d\mu \hbox{ , } \varphi \leq f \hbox{ , } \varphi \text{ non-negative simple function} \}
\]
This definition give the monotonicity, the additivity and the positive homogeneity of the Lebesgue integration, propreties deduced directly from the simple functions.

\paragraph{Monotone convergences} Let $f_n$ be a sequence of measurable functions, non-negative, non-decreasing ($0 \leq f_n \leq f_{n+1}$), then
\[
\int \underset{n}{\text{lim}} f_n d\mu = \underset{n}{\text{lim}} \int  f_n d\mu 
\]

\paragraph{Fatou's lemma}
Let $f_n$ be a sequence of $\textbf{non-negative}$ measurable functions, then
\[
\int \underset{n}{\varliminf} f_n d\mu \leq \underset{n}{\varliminf} \int  f_n d\mu 
\]

\paragraph{Dominated convergences}
Let $f_n$ a sequence of measurable functions, satisfying 
\begin{itemize}
 \item $f_n \longrightarrow f $ pointwise almost everywhere
 \item $f_n \leq g \hbox{ , } \forall n$ where $g$ is integrable 
\end{itemize}
then
\[
\int \underset{n}{\text{lim}} f_n d\mu = \underset{n}{\text{lim}} \int  f_n d\mu 
\hspace{1cm}
\text{or equivalently}
\hspace{1cm}
 \underset{n}{\text{lim}} \int |f_n -f| d\mu =0
\]

\paragraph{Jensen inequality} Let $X$ an integrable random variable and $\varPhi$ a convexe function such that $\varPhi(X)$ is also integrable. Then 
\[
\varPhi(\mathbb{E}[X]) \leq \mathbb{E}[\varPhi(X)]
\]

\subsection{\texorpdfstring{$\textbf{L}^p$}{LpSpace} space}
\paragraph{Holder inequality} Let $p,q \in [1,\infty]$ which are Holder conjugates $\frac{1}{p} + \frac{1}{p} = 1$, then
\[
\| f.g \|_{1} \leq \| f \|_{p}  \| g \|_{q}  
\]
If $p,p \in (1,\infty)$ and $ \| f \|_{p} + \| g \|_{q} <  \infty$ then there are equality iff $\alpha |f|^p = \beta |g|^q$ $\mu$-a.e

\paragraph{Minkowski inequality} Let $p \in [1,\infty]$ and $f,g \in \textbf{L}^p$, then
\[
\| f+g \|_p \leq \| f \|_p  + \| g \|_p 
\]
The case $p=1 \text{ or } p=\infty $ follow from the triangular inequality. In case  $p \in (1,\infty)$, equality happens iif $f=0 \text{ a.e }$ or $ g= \alpha.f \text{ a.e}$. The Minkowski inequality can be generalized for a sequence $f_{n}$ of $\textbf{non-negative}$ functions in $\text{L}^p$ :
\[
\| \sum_{n \geq 1} f_n \|_p \leq  \sum_{n \geq 1} \| f_n \|_p
\]

\paragraph{Densities in $\textbf{L}^p$ spaces with $1 \leq p < \infty$}
For the first view on the $\mathbb{R}^n$ Lebesgue measure which is \textbf{regular}, the densities for $\textbf{L}^p$ spaces with $1 \leq p < \infty$ are built step by step :
\begin{itemize}
 \item \textbf{Simple integrable functions} are $\|.\|_p$-dense in $\text{L}^p \hbox{ , } 1 \leq p < \infty$ (Beppo-Levi)
 \item \textbf{Steps functions with compact support} are $\|.\|_p$-dense in \textbf{Simple integrable functions} (use of inner regularity)
 \item \textbf{Continuous functions with comptact support} are $\|.\|_p$-dense in \textbf{Steps functions with comptact support} (distance function to a set)
\end{itemize}
By the implication of densities, we conclude that all are dense in $\textbf{L}^p_{1 \leq p < \infty}$ spaces for the norm $\|.\|_p$.
 

\paragraph{$\textbf{L}^{\infty}$ space} si firstly defined by the essential supremum of a non-negative measurable function  :
\[
\text{supess} f = \text{inf} \{a > 0 | \mu(\{ f>a \}) = 0 \}
\]
Therefor we can define a equivalent class of bounded functions with the essential bounded functions with the $\mu$-a.e. Then all the Banach propreties of the norm $\|.\|_{\text{sup}}$ imply the one of the norm $\|.\|_{\infty}$. The space $\text{L}^{\infty}$ has almost the same principale propreties of the other spaces $\textbf{L}^p_{1 \leq p < \infty}$, except the important difference is that the last one is separable while the first one is not (density of step functions).

\paragraph{Venn diagram illustrating $\textbf{L}^{p}$ space}
\begin{center}
\begin{pspicture}(-4,-4)(4,4)
%\psline(-4,-4)(4,4)
\psaxes[labels=none,ticks=none]{->}(0,0)(-4,-4)(4,4)
\psline[linewidth=0.2pt](3,0)(0,3)(-3,0)(0,-3)(3,0)
\psframe[linestyle=dashed,linewidth=0.2pt,fillstyle=solid,fillcolor=lightgray](-2,-1)(2,1)
\rput(3.1, 0.2){\Rnode{One}{1}}
\rput(2.1, 0.3){\Rnode{P}{$\frac{1}{p}$}}
\rput(0.2, 1.3){\Rnode{Q}{$\frac{1}{q}$}}
\rput(3.2, 1.2){\Rnode{PQ}{$\frac{1}{p}+\frac{1}{q}=1$}}
\rput(0, 0){\Rnode{Lp}{$\text{L}^p$ space}}
\end{pspicture}
\end{center}
The $\text{L}^p$-spaces have myriad relationshisp that can sometimes be confusing to remember. Observe that all the $\text{L}^p$-spaces live inside the square, that no $\text{L}^p$-space is contianed in the other. The space $\text{L}^{\infty}$ is the vertical line and the space $\text{L}^1$ is the horizontal line. The space $\text{L}^2$ is a square, which indicates self-duality. The spaces $\text{L}^p$ and $\text{L}^q$ are congruent rectangles: we can get one from the other geometrically by reflecting in the line $y = x$, which indicate that they are dual. Finally, the intersection of $\text{L}^p$ and $\text{L}^r$ is contained in all the $\text{L}^s$ for $p \leq s \leq r$, a fact again reflected in the Venn diagram.

\subsection{Randon-Nykodym}
\paragraph{Theorem } Let $\mu$ and $\nu$ two $\sigma$-\textit{finite} (or simply finite) measure on $(X,\mathcal{A})$, then there are equivalent
\begin{itemize}
 \item $\forall A \in \mathcal{A} \hbox{ , } \mu(A)=0 \Longrightarrow \nu(A)=0$ (absolute continuity $\nu \ll \mu$)
 \item $\exists f:X\longmapsto \mathbb{R}_{+} \text{ such that } \forall A\in \mathcal{A} \hbox{ , } \nu(A)= \int_A fd\mu $
\end{itemize}
This function is $\mu$-a.e uniquely defined, called the derivative $f=\frac{d\nu}{d\mu}$ 
\paragraph{Propreties }
 Let $\nu$, $\mu$ and $\lambda$ be $\sigma$-finite measure.
\begin{enumerate}
 \item If $\nu \ll \lambda$ and $\mu \ll \lambda$, then
\[
\frac{d(\nu+\mu)}{d\lambda} = \frac{d\nu}{d\lambda} + \frac{d\mu}{d\lambda} \hspace{5mm} \lambda\text{-a.e}
\] 
 \item If $\nu \ll \mu \ll \lambda$, then
\[
\frac{d\nu}{d\lambda} = \frac{d\nu}{d\mu} \frac{d\mu}{d\lambda} \hspace{5mm} \lambda\text{-a.e}
\]
 \item If $\mu \ll \lambda$ and $g$ is a $\mu$-integrable function, then
\[
\int_{X} gd\mu = \int_{X} g .\frac{d\mu}{d\lambda} .d\lambda
\] 
\end{enumerate}

\subsection{Fubini's theorem}
\begin{description}
 \item[Fubini-Tonelli] \hfill \\ 
 Let $(X,\mathcal{A},\mu)$ and $(Y,\mathcal{B},\nu)$ be two $\sigma$-finite measure spaces, let $f:(X\times Y, \mathcal{A} \otimes \mathcal{B})\longrightarrow \bar{\mathbb{R}}_+$ a measureable \textit{non-negative} function. Then
 \begin{enumerate}[i]
  \item The functions everywhere defined $x\longmapsto \int_{Y}f(x,y)d\nu_y$ and $y\longmapsto \int_{X}f(x,y)d\mu_x$ are respectively $\mathcal{A}$ and $\mathcal{B}$-measurable.
  \item $\int_{X\times Y} fd\mu \otimes \nu = \int_X (\int_{Y}f(x,y)d\nu_y )d\mu_x = \int_Y(\int_{X}f(x,y)d\mu_x)d\nu_y$
 \end{enumerate} 
 \item[Fubini-Lebesgue] \hfill \\
 Let $(X,\mathcal{A},\mu)$ and $(Y,\mathcal{B},\nu)$ be two measure spaces, let $f \in \mathcal{L}^1(\mu \otimes \nu)$ a \textit{integrable} function on the product measure. Then
  \begin{enumerate}[i]
   \item $
	  \left\{
	  \begin{array}{l}
	  \mu(dx)\text{-a.e } \hspace{5mm} y\longmapsto f(\cdot,y) \in \mathcal{L}^1(\nu)   \\
	  \nu(dy)\text{-a.e } \hspace{5mm} x\longmapsto f(x,\cdot) \in \mathcal{L}^1(\mu)  
	  \end{array}
	  \right.
	 $
   \item $
	  \left\{
	  \begin{array}{l}
	  x\longmapsto \int_Y f(x,y)d\nu_y \in \mathcal{L}^1(\mu)  \text{ is a function defined } \mu(dx)\text{-a.e on } X\\
	  y\longmapsto \int_X f(x,y)d\mu_x \in \mathcal{L}^1(\nu)  \text{ is a function defined } \nu(dy)\text{-a.e on } Y
	  \end{array}
	  \right.
	 $
   \item $\int_{X\times Y} fd\mu \otimes \nu = \int_X (\int_{Y}f(x,y)d\nu_y )d\mu_x = \int_Y(\int_{X}f(x,y)d\mu_x)d\nu_y$
  \end{enumerate}

 \end{description}
 
\subsection{Push-forward measure} 

\paragraph{Definition-theorem} Let $(X,\mathcal{A})$ and $(Y,\mathcal{B})$ two measurable spaces, $h:X\longrightarrow Y$ a measurable mapping, $\mu$ a positive measure on $(X,\mathcal{A})$. The function noted $\mu_h$ or $h_{*}\mu$ defined by 
\[
h_{*}\mu : \mathcal{B} \longrightarrow \bar{\mathbb{R}}_+  \hspace{10mm} h_{*}\mu(B) = \mu(h^{-1}(B))     \hspace{5mm} \forall B \in \mathcal{B}
\]
is a measure on $(Y,\mathcal{B})$ (to prove) and is called pushforward measure. 

\paragraph{Transfert theorem} Let things be defined as above and a measureable function $f:Y\longrightarrow \mathbb{K}$. Then
\begin{itemize}
\renewcommand{\labelitemi}{$\vcenter{\hbox{\tiny$\bullet$}}$}
 \item if $f \geq 0$ then $\int_Y f d(h_{*}\mu) = \int_X f\circ h d\mu$
 \item $f \text{ is } (h_{*}\mu)\text{-integrable} \Longleftrightarrow f\circ h \text{ is } \mu\text{-integrable}$ in this case we also have
 \[\int_Y f d(h_{*}\mu) = \int_X f\circ h d\mu\]
\end{itemize}

\paragraph{Change of variables theorem}
Let $\varDelta$ and $D$ be two open sets of $\mathbb{R}^n$ and $\Phi:\varDelta \longrightarrow D$ a $\mathcal{C}^1$-diffeomorphism. Then these three statements below are equivalent :
\begin{itemize}
 \item The Lebesgue measure on $D$ is the $\Phi$-pushforward measure of $|J_{\Phi}|.\lambda_{\varDelta}$ (measure on $\varDelta$ by density $|J_{\Phi}|$), i.e  $\lambda_{D} = \Phi_{*}(|J_{\Phi}|.\lambda_{\varDelta})$
 \item For every borelian function $f : D \longrightarrow \mathbb{R}_+$,
 \[
 \int_D f(x)dx = \int_{\varDelta} f(\Phi(u)) |J_{\Phi}|(u) du \leq \infty
 \]
 \item For every borelian function $f : D \longrightarrow \mathbb{R}_+$, f is $\lambda_{D}$-integrable  if and only if $(f\circ \Phi) |J_{\Phi}|$ is $\lambda_{\varDelta}$-integrable and in this case, we also have 
  \[
 \int_D f(x)dx = \int_{\varDelta} f(\Phi(u)) |J_{\Phi}|(u) du
 \]
\end{itemize}
Where $J_{\Phi}(x)= \text{det} \frac{d\Phi}{dx} $

\section{Probability}
\subsection{Uniform integrability}
Let $(X_t)_{t\in T}$ a family of $L^1$ r.v. Then there are two equivalent definitions of uniform integrability. We say this family is uniformly integrable iff
\paragraph{Definition 1}
\begin{enumerate}[i]
 \item $\sup_{t\in T} \mathbb{E}[ X_t ] < \infty$
 \item for all $\varepsilon >0$, there exist $\delta>0$ such that for any event $A$,
 \[   \mathbb{P}(A) <\delta \Longrightarrow \sup_{t\in T} \mathbb{E}[|X_t|\mathbf{1}_{A}] \leq \varepsilon \]
\end{enumerate}
\paragraph{Definition 2}
\[
\lim_{a\rightarrow \infty} \sup_{t\in T} \mathbb{E}[|X_t|\mathbf{1}_{|X_t|>a}] = 0
\]
Usually, when need to prove if a family is uniformly integrable, we use the second definition, but when knowing a family is already u.i, we use the first definition to use.


\subsection{Independence and conditional}
\paragraph{Independence}
\begin{itemize}
\renewcommand{\labelitemi}{$\vcenter{\hbox{\tiny$\bullet$}}$}
 \item \textbf{Def} The notion of independance is related to a measure. Two events $A$ and $B$ is independant iff 
 \[
  \mathbb{P}(A \cap B ) = \mathbb{P}(A). \mathbb{P}(B) 
 \]
 \item \textbf{Def} Two familly of events $\mathcal{A}_1$ and $\mathcal{A}_2$ are independant iff each event in $\mathcal{A}_1$ is independant to each event in $\mathcal{A}_2$
 \item \textbf{Def : } Two random variables $X_1$ and $X_2$ are independant iff  $\sigma(X_1)$ and $\sigma(X_2)$ are independant.
 \item \textbf{Theorem } Let $(\Omega,\mathcal{A},\mathbb{P})$ be a probability space, let $\mathcal{C}_1$ and $\mathcal{C}_2$ be two $\pi$-systems in $\mathcal{A}$. Then $\sigma(\mathcal{C}_1)$ is independant to $\sigma(\mathcal{C}_2)$ if and only if $\mathcal{C}_1$ is independant to $\mathcal{C}_2$
 \item \textbf{Theorem } Two random variables $X_1$ and $X_2$ are independant if and only if 
 \[
 \mathbb{P}_{X_1X_2} = \mathbb{P}_{X_1} \otimes \mathbb{P}_{X_2}
 \]
 \item \textbf{Theorem } Let $X_1$ and $X_2$ be two random variables. Then the functional version of independance state that :
 \begin{enumerate}[i]
  \item $X_1$ and $X_2$ are independant.
  \item For every measurable \textit{positive} real functions $f_1$ and $f_2$ we have 
  \[
    \int_{\Omega} f_1 \circ X_1 . f_2 \circ X_2 d\mathbb{P} = \int_{\Omega} f_1 \circ X_1 d\mathbb{P} . \int_{\Omega} f_2 \circ X_2 d\mathbb{P} 
  \]
  For every measurable \textit{bounded} real functions $f_1$ and $f_2$ we have 
  \[
    \int_{\Omega} f_1 \circ X_1 . f_2 \circ X_2 d\mathbb{P} = \int_{\Omega} f_1 \circ X_1 d\mathbb{P} . \int_{\Omega} f_2 \circ X_2 d\mathbb{P} 
  \]
 \end{enumerate}
 \item \textbf{Theorem } Let $X,Y$ two \textit{independent} random variables to $(E_1,\mathcal{E}_1)$ and $(E_2,\mathcal{E}_2)$, Then for every measurable positive or integrable function $f : E_1\times E_2 \longrightarrow \mathbb{R}$, we have
 \[
 \mathbb{E}[f(X,Y)] = \int_{\Omega}\int_{\Omega} f(X(\omega),Y(\omega')) d\mathbb{P}(\omega)d\mathbb{P}(\omega)
                    = \int_{\Omega}\int_{\Omega} f(x,y) d\mathbb{P}_{X}d\mathbb{P}_{Y}
 \]
\end{itemize}
\paragraph{Asymptotical $\sigma$-algebra} Let $(\mathcal{A}_n)_{n\in\mathbb{N}}$ a sequence of $\sigma$-algebra in $\mathcal{A}$. Note that $\bigvee_{n}\mathcal{A}_n$ is the $\sigma$-algebra generated by $\cup_n \mathcal{A}_n$. Then the asymptotic $\sigma$-algebra is defined by 
\[
\mathcal{A}_{\infty} = \bigcap_{n \geq 0}(\bigvee_{p\geq n} \mathcal{A}_p) 
\]
The event in this $\sigma$-algebra is called \textit{tail event}. For a sequence of random variables $(X_n)_{n\in \mathbb{N}}$, it is the same definition :
\[
\mathcal{A}_{\infty} = \bigcap_{n \geq 0} \sigma( X_p | {p\geq n}) 
\]
\paragraph{0-1 law} Let $(\mathcal{A}_n)_{n\in\mathbb{N}}$ a sequence of independent sub $\sigma$-algebras contained in $\mathcal{A}$, then for every tail event $A \in \mathcal{A}_{\infty}$, we have $\mathbb{P}(A) = 0 \text{ or }  \mathbb{P}(A) = 0  $ 
\paragraph{Borel-Cantelli lemma } Let $(A_n)_{n\in\mathbb{N}}$ be a sequence of events. Then 
\begin{enumerate}
 \item 
 \[
  \sum_{n=0}^{\infty} \mathbb{P}(A_n) <\infty  \Longrightarrow \mathbb{P}(\underset{n}{\varlimsup} A_n) =0
 \]
 \item If $A_n$ are independent then 
 \[
   \sum_{n=0}^{\infty} \mathbb{P}(A_n) = \infty  \Longrightarrow \mathbb{P}(\underset{n}{\varlimsup} A_n) =1
 \]
\end{enumerate}

\subsection{Conditional probability}
\paragraph{Definition of transition probabilities } Let $(E,\mathcal{E})$ and $(F,\mathcal{F})$ be two probabilisable spaces. A transition probability $\nu$ is defined as below
\begin{enumerate}
 \item $\nu : E\times \mathcal{F} \longrightarrow [0,1] $
 \item $\forall x\in E, \hspace{5mm} \nu(x,.)$ is a probability on $(F,\mathcal{F})$
 \item $\forall B\in \mathcal{F}, \hspace{5mm} \nu(.,B)$ is a $\mathcal{E}$-mesurable function.
\end{enumerate}
\textbf{Jirina's theorem} state that if $X,Y$ two random variable to polish spaces (metric complete separable), then there existe the transition probability.
\paragraph{Propreties}
By giving further a probability $\lambda$ on $(E,\mathcal{E})$, we define a product measure on $(E\times F, \mathcal{E} \otimes \mathcal{F})$ noted $\lambda \cdot \nu$ :
\[
  \forall A\times B \in \mathcal{E}\times \mathcal{F}, \hspace{5mm} \lambda \cdot \nu(A\times B) = \int_A \nu(x,B)d\lambda(x)
\]
Then for a measurable function $f : E\times F, \mathcal{E}\otimes \mathcal{F} \longrightarrow  \bar{\mathbb{R}}$, positive or $\lambda\cdot\nu$-integrable :
\[
\int_{E\times F} f d\lambda \cdot \nu =
\int_E [ \int_F f(x,y) \nu(x,dy)  ] d\lambda(x)
\]
For a given $x$, if $f:F\longrightarrow \mathbb{R}$ is a positive or $\nu(x,.)$-integrable function, we note $\nu(x,f)$ or $\nu f(x)$ the quantity $\int_F f(y) \nu(x,dy) $ which is a function of $x$. We can also define an ''average'' measure $\mu$ on $(F,\mathcal{F})$ by
\[
 \forall B \in \mathcal{F}, \hspace{5mm} \mu(B) = \int_E \nu(x,B)d\lambda (x)
\]
For all function $g:F\longrightarrow \mathbb{R}$, positive or $\mu$-integrable, the map $\nu(\cdot, g)$ is defined $\lambda$-a.e, and 
\[
\int_F gd\mu = \int_E \nu(x,g) d\lambda(x)
\]
\paragraph{For random variables} Let $X,Y$ be two random variables with state spaces $E,F$. Then the conditional probability of $Y$ given $X$ is the transition probability noted $\mathbb{P}^{X=\cdot}_Y$ \textit{satisfying} 
\[
\mathbb{P}_{X,Y} =\mathbb{P}_X\cdot \mathbb{P}^{X=\cdot}_Y
\]
\paragraph{Conditional transfert theorem} Let $X,Y$ be two random variables with state spaces $E,F$. Assuming there existe a conditional probability of $Y$ given $X$ noted $\nu= \mathbb{P}^{X=\cdot}_Y$. If futher giving a $(E\times F, \mathcal{E}\otimes \mathcal{F})$-measurable function $f$, then we have the conditional probability of $f(X,Y)$ given $X$ noted $\mu$ satisfy :
\[
\forall x\in E, \hspace{5mm} \mu(x,\cdot) = f(x,\cdot)[\nu(x,\cdot)] \hspace{5mm} \text{ also noted } \hspace{5mm}
\mathbb{P}^{X=x}_{f(X,Y)} = \mathbb{P}^{X=x}_{f(x,Y)}
\]
\paragraph{Conditional moments} Let $X,Y$ be two random variables with state spaces $E,F$. Assuming there existe a conditional probability of $Y$ given $X$, and $Y \in L^p , p \in \mathbb{N}^*$, then $\mathbb{P}_X$-a.s we can define the conditional $p$-moment $\int_F |y|^p.  d\mathbb{P}^{X=\cdot}_Y(y)<\infty$. The first moment is noted $m^{X=\cdot}_Y$, and noticed that it is not exactly the conditional expectation. 


\subsection{Conditional expectation}
\paragraph{Definitions } There are two equivalent definition versions of conditional expectatioin : the measure approach and the geometrical approach :
\begin{description}
 \item [Measure approach] Let $(\Omega, \mathcal{A}, \mathbb{P})$ be a probability space, with a sub-$\sigma$-algebra $\mathcal{B} \subset \mathcal{A}$, and $X$ a random variable. Then a conditional expectation fo $X$ given $\mathcal{B}$ is a $\mathcal{B}$-measurable function noted $\mathbb{E}(X|\mathcal{B})$ satisfying :
 \[
  \forall B \in \mathcal{B}, \hspace{1cm}  \int_B \mathbb{E}(X|\mathcal{B}) d\mathbb{P} = \int_B Xd\mathbb{P}
 \]
 This function is a.s unique. 
 \item [Geometry approach]Let $(\Omega, \mathcal{A}, \mathbb{P})$ be a probability space, with a sub-$\sigma$-algebra $\mathcal{B} \subset \mathcal{A}$. Then the subspace $L^2(\Omega, \mathcal{B}, \mathbb{P})$ is closed in the Hilbert space $L^2(\Omega, \mathcal{A}, \mathbb{P})$. Then we can define a orthogonal projection from $L^2(\Omega, \mathcal{A}, \mathbb{P})$ to  $L^2(\Omega, \mathcal{B}, \mathbb{P})$ noted $\mathbb{E}(.|\mathcal{B})$. That for any $X \in L^2(\Omega, \mathcal{A}, \mathbb{P})$, we note its projectee $\mathbb{E}(X|\mathcal{B})$, caracterized by 
 \[
 \forall Z \in L^2(\Omega, \mathcal{B}, \mathbb{P}), \hspace{1cm}  \mathbb{E}(ZX) = \mathbb{E}[Z\mathbb{E}(X|\mathcal{B})]
 \] 
\end{description}
For two random variable $X,Y$, if the conditional expectation exist, $\mathbb{E}[Y|X]$ is a function of $X$

\paragraph{Propreties } Thanks to the equivalence of two definition approachs, the propreties of conditional expectation can be proved by one way or other. Note that on a probability space, $L^2(\Omega, \mathcal{A}, \mathbb{P}) \in L^1(\Omega, \mathcal{A}, \mathbb{P})$.
\begin{enumerate}
 \item $\mathbb{E}[\mathbb{E}(X|\mathcal{B})] = \mathbb{E}(X)$
 \item If $\sigma(X)$ and $\mathcal{B}$ are independent, then $\mathbb{E}(X|\mathcal{B}) = \mathbb{E}(X)$ a.e
 \item On $L^2(\Omega, \mathcal{A}, \mathbb{P})$ or $L^1(\Omega, \mathcal{A}, \mathbb{P})$ the operator $\mathbb{E}(.|\mathcal{B})$ is a positive, monotonic, contractive, linear continuous.
 \begin{description}
  \item [Linear] $\mathbb{E}(\alpha X + Y|\mathcal{B})$ = $\alpha \mathbb{E}(X|\mathcal{B})$ + $\mathbb{E}(Y|\mathcal{B})$
  \item [Positive] $X \geq 0 \Longrightarrow \mathbb{E}(X|\mathcal{B}) \geq 0 $ 
  \item [Monotonicity] $X \geq Y \Longrightarrow \mathbb{E}(X|\mathcal{B}) \geq \mathbb{E}(Y|\mathcal{B})$
  \item [Contractivity] $\| \mathbb{E}(X|\mathcal{B}) \|_{L^p} \leq \| X \|_{L^p} \hspace{1cm} \forall p \geq 1$
  \item [Continuity] is implied from the contractivity. 
 \end{description}
  \item If $X$ is $\mathcal{B}$-measurable, then $ \mathbb{E}(X|\mathcal{B}) = X$
  \item If $Z$ is bounded $\mathcal{B}$-measurable, then $ \mathbb{E}(ZX|\mathcal{B}) = Z\mathbb{E}(X|\mathcal{B})$ a.e
  \item If $\mathcal{B}_1 \subset \mathcal{B}_2 \subset \mathcal{A}$, then $\mathbb{E}[ \mathbb{E}(X|\mathcal{B}_2) | \mathcal{B}_1] = \mathbb{E}(X|\mathcal{B}_1)$
  \item $\mathbb{E}(.|\mathcal{B})$ can view as an ''integration'' and then has all of integration propreties, i.e :
  \begin{description}
   \item [Triangular inequality] $|\mathbb{E}(X/\mathcal{B})| \leq \mathbb{E}(|X|/\mathcal{B})$
   \item [Monotonic convergence] For a non-negative non-decreasing sequence $X_n$ converging a.e to $X$
   \[
   \mathbb{E}(X|\mathcal{B}) = \mathbb{E}(\lim_n \uparrow X_n|\mathcal{B}) = \lim_n \uparrow \mathbb{E}(X_n|\mathcal{B})
   \]
   \item [Fatou's lemma]  For a non-negative sequence $X_n$
   \[
   \mathbb{E}(\varliminf_n  X_n|\mathcal{B}) \leq \varliminf_n  \mathbb{E}(X_n|\mathcal{B})
   \]
   \item [Dominated convergence] Let $X_n$ be a a.e convergent sequence $X_n \xrightarrow{a.s} X$ and $Y$ a $L^1$ dominator, i.e $Y\in L^1(\Omega, \mathcal{A}, \mathbb{P}), \hspace{3mm} |X_n| \leq Y,\hspace{3mm} \forall n$, then
   \[
   \mathbb{E}(X|\mathcal{B}) = \mathbb{E}(\lim_n  X_n|\mathcal{B}) = \lim_n \mathbb{E}(X_n|\mathcal{B}) \text{  a.e}
   \]
   \item [Jensen inequality] If $\Phi$ is a convexe function that $\Phi(X)$ is also in $L^1(\Omega, \mathcal{A}, \mathbb{P})$, then 
   \[
   \Phi( \mathbb{E}[X|\mathcal{B}] ) \leq \mathbb{E}[\Phi(X)|\mathcal{B}]
   \]
    \item [Product measure] Let $X,Y$ two \textit{independent} random variables to $(E_1,\mathcal{E}_1)$ and $(E_2,\mathcal{E}_2)$, with $X$ is $\mathcal{G}$-measurable. Then for every measurable positive or integrable function $\varphi : E_1\times E_2 \longrightarrow \mathbb{R}$, we have
    \[
    \mathbb{E}[\varphi(X,Y)|\mathcal{G}](\omega)  = \int_{\omega' \in \Omega} \varphi(X(\omega),Y(\omega ')) d\mathbb{P}(\omega ')
    \]
    This function is noted $\mathbb{E}[\varphi(x,Y)]$ for $x=X(\omega)$
  
  \end{description}

\end{enumerate}

\paragraph{Link with the conditional probability } Let $X,Y$ two random variables, $Y\in L^1$ and there exist the conditional probability of $Y$ given $X$ : $\mathbb{P}^{X=\cdot}_Y$, then we have the relation below
\[
\mathbb{E}[Y|X] = \mathbb{E}[Y|\sigma(X)] = m^{X=\cdot}_Y \circ X = (\int_F |y| d\mathbb{P}^{X=\cdot}_Y(y) )\circ X
\]

\subsection{Conditional independance}

 
 
 
 
 
 
 
 
 
 
 
 
\section{Functional analysis}
\subsection{Normed vector space} 
\paragraph{Finite dimension}
\begin{itemize}
 \item On a normed vector space of finite dimension, every norms are equivalent.
 \item A normed vector space of finite dimension is complete.
 \item Finite product of complete normed vector space is complete.
\end{itemize}

\paragraph{Continuous map}
\begin{itemize}
 \item Let $E$ , $F$ be two normed vector spaces on the scalar field $\mathbb{K} = \mathbb{R} \text{ or } \mathbb{C} $, and $f:E\longrightarrow F$ a linear map. Then these above are equivalent :
 \begin{enumerate}
  \item $f$ is continuous at one point $x_0$ of $E$
  \item $f$ is continuous
  \item $f$ is uniformly continuous
  \item $f$ is bounded
 \end{enumerate}
 When $f$ is continuous, we can define the norm : $\|f\| = \sup_{x\in E, \|x\|\leq 1} \|f(x)\|_{F}$. The space $\mathcal{L}(E,F)$ of continuous linear map is a normed vector space with this norm. If $F$ is complete, $\mathcal{L}(E,F)$ is complete too.
 \item Let $E$, $F$ and $G$ be three normed vector space, and $f \in \mathcal{L}(E,F)$, $g \in \mathcal{L}(F,G)$, then
 \[ g\circ f \in \mathcal{L}(E,G) \hspace{5mm} \text{ and } \hspace{5mm} \| g\circ f\| \leq \|g\|.\|f\| \]
 \item All linear map from a \textit{finite dimension} N.V.S to a arbitrary normed vector space are continuous.
 \item The topological dual $E'$ of all \textit{continuous} linear form (linear functional) is a complete space.
\end{itemize}
\paragraph{Hahn-Banach's theorem} 
\begin{itemize}
\renewcommand{\labelitemi}{$\vcenter{\hbox{\tiny$\bullet$}}$}
 \item Let $V$ be a vector space over $\mathbb{R}$, a linear subspace $U \subset V$.
 \item Let $p:V\longrightarrow \mathbb{R}$ a sublinear function.
 \item Let $\phi:U\longrightarrow \mathbb{R}$ a linear functional, which is dominated by $p$ : $\phi(x)\leq p(x) \hspace{2mm} \forall x \in U$
\end{itemize}
then there exists a linear functional $\psi:V\longrightarrow \mathbb{R}$ such that extend $\phi$, i.e
\begin{itemize}
\renewcommand{\labelitemi}{$\vcenter{\hbox{\tiny$\bullet$}}$}
 \item $\psi_{|U}=\phi$
 \item $|\psi(x)| \leq p(x) \hspace{2mm} \forall x \in V $
\end{itemize}

\subsection{Banach space}
Let $X$ a normed vector space. There are two equivalent definition of Banach space
\begin{itemize}
 \item $X$ is a complete space.
 \item Each absolutely convergent series in $X$ converges : $\sum_n \|v_n\| < \infty \Longrightarrow \sum_n v_n \text{ converges }  $ 
\end{itemize}
If $Y$ is a Banach space, then $\mathcal{L(X,Y)}$ is Banach too.


\subsection{Hilbert space ...}%TODO


\subsection{Other theorems}

\paragraph{Banach fixed-point theorem}
Let $(X,d)$ be a non-empty complete metric space with a contraction mapping $T:X\longrightarrow X$. Then $T$ admits a unique fixed-point $x^{*}$ in $X$ satisfying $T(x^{*}) = x^{*}$. Furthermore, $x^{*}$ can be found as followsm star with an arbitrary element $x_0$ in $X$ and define a sequence ${x_n}$ by $x_{n+1}=T(x_n)$, then $x_n \longrightarrow x^*$

\paragraph{Hilbert projection theorem} Let $H$ be a Hilbert space, and a \textit{closed convex} $C \subset H$. Then for every $x\in H$, there exists a unique point $x^{*}\in C$ that minimized $\|x-x^{*}\|_{H}$ over $C$. In particular, if $C$ is a closed subspace of $H$, the necessary and sufficient condition for finding $x^{*}$ is that $(x-x^{*}) \perp C$


\paragraph{Implicit function theorem...}%TODO
\paragraph{Lagrange multiplier...}%TODO
\paragraph{Local inversion theorem..}%TODO

\paragraph{Lax–Milgram theorem} Let $b$ be a bounded coercive bilinear functional on a Hilbert space $H$. Then for every bounded linear functional $L$ on $H$ there exists a unique $u_{L} \in H$ such that 
\[
L(v) = b(v,u_{L}) \hspace{2cm} ,\forall v \in H
\]

\paragraph{Riesz representation theorem...}%TODO
\paragraph{Riesz–Markov–Kakutani representation theorem...}%TODO
\subsection{densities in functional spaces}
\paragraph{Density $\text{Lip}_b(X) \text{ in } \textbf{L}^p_{1 \leq p < \infty}$} Let $X$ be a metric space and $\mu$ an \textit{outer regular} measure on its borelians. Let $\text{Lip}_b(X)$ be the set of all bounded lipschitz functions (to $\mathbb{R} \text{ or } \mathbb{C} $). Then for all $ p \in [1, \infty[ $
\[
\text{Lip}_b(X) \cap \mathcal{L}^p(\mu) \text{ is } \|.\|_p \text{-dense in }  \mathcal{L}^p(\mu)
\]
\paragraph{Density $\text{Lip}_K(X) \text{ in } \textbf{L}^p_{1 \leq p < \infty}$} be a metric space, \textit{separable and locally compact}. Let $\mu$ be a \textit{Borel measure} on such a metric space (imply regular measure). Let $\text{Lip}_{K}(X)$ be the set of all compactly supported lipschitz functions (to $\mathbb{R} \text{ or } \mathbb{C} $). Then for all $ p \in [1, \infty[ $
\[
\text{Lip}_{K}(X) \text{ is } \|.\|_p \text{-dense in }  \mathcal{L}^p(\mu)
\]




\section{Differential equations ...}%TODO

\end{document}