\documentclass[]{beamer}
% Class options include: notes, notesonly, handout, trans,
%                        hidesubsections, shadesubsections,
%                        inrow, blue, red, grey, brown

% Theme for beamer presentation.
\usepackage{beamerthemesplit} 
\usepackage{pstricks}
\usepackage{pst-node}

\title{Pygmalion Uncertainty Analysis}
\author{NGUYEN Chi-Thanh} 
\institute{MAS Central SA}
\date{\today}             

\begin{document}

% Creates title page of slide show using above information
\begin{frame}
  \titlepage 
\end{frame}


% Creates table of contents slide incorporating
% all \section and \subsection commands
\begin{frame}
  \tableofcontents[hideallsubsections]
\end{frame}

%%%%%%%%%%%%%%%%%%%%%%%%%%%%%%%%%%%%%%%%%%%%%%%%%%%%%%%%%%
\section{Pygmalion in 10s}
\begin{frame}
  \frametitle{Pygmalion - Global Design}
  \begin{columns}[c]
  \column{2in}  % slides are 3in high by 5in wide
    \begin{itemize} 
    \item Models
    \item Functionalities
    \end{itemize}
  \column{3in}
    \framebox{UML Image}
  \end{columns} 
\end{frame}
%%%%%%%%%%%%%%%%%%%%%%%%%%%%%%%%%%%%%%%%%%%%%%%%%%%%%%%%%%
\section{pgmaUA - Uncertainty Analisis}
%%%%%%%%%%%%%%%%%%%%
\begin{frame}
  \frametitle{pgma::UA - global idea}
  \begin{itemize}
  \item<1-> Global idea of Uncertainty Analysis
  \item<2-> Inputs
  \item<3-> Outputs
  \end{itemize}
\only<2>{
  \begin{center}
  \begin{pspicture}(-4,-1.5)(4,1.5)
    \psline(-3,-1.5)(3,1.5)
  \end{pspicture}
  \end{center}
}
\end{frame}
%%%%%%%%%%%%%%%%%%%%
\begin{frame}
  \frametitle{pgma::UA}
  \begin{itemize}
  \item MonteCarlos
  \item Unscented Transform
  \end{itemize}
\end{frame}
%%%%%%%%%%%%%%%%%%%%%%%%%%%%%%%%%%%%%%%%%%%%%%%%%%%%%%%%%%
\subsection{pgmaUA::MonteCarlos}
%%%%%%%%%%%%%%%%%%%%
\begin{frame}
  \frametitle{MonteCarlos MC}
  \begin{itemize}
  \item Input
  \item Output
  \end{itemize}
\end{frame}
%%%%%%%%%%%%%%%%%%%%
\begin{frame}
  \frametitle{Example results}
  \begin{itemize}
  \item Example with Loka-Volterra
  \end{itemize}
\end{frame}
%%%%%%%%%%%%%%%%%%%%%%%%%%%%%%%%%%%%%%%%%%%%%%%%%%%%%%%%%%
\subsection{pgmaUA::UnscentedTransform}
%%%%%%%%%%%%%%%%%%%%
\begin{frame}
  \frametitle{Unscented Transform}
  \begin{itemize}
  \item Input
  \item Output
  \end{itemize}
\end{frame}
%%%%%%%%%%%%%%%%%%%%
\begin{frame}
  \frametitle{Example results UT}
  \begin{itemize}
  \item Example with Loka-Volterra
  \end{itemize}
\end{frame}
\subsection{pgmaUA comparison MC-UT}
%%%%%%%%%%%%%%%%%%%%
\begin{frame}
  \frametitle{MonteCarlos vs UnscentedTransform}
  \begin{itemize}
  \item speed
  \item nonlinearities
  \end{itemize}
\end{frame}

%%%%%%%%%%%%%%%%%%%%%%%%%%%%%%%%%%%%%%%%%%%%%%%%%%%%%%%%%%
\section{Unscented Kalman Filter}
%%%%%%%%%%%%%%%%%%%%
\begin{frame}
  \frametitle{Global Idea of Filter}
  \begin{itemize}
  \item Inputs (Experimental Observations)
  \item Output (Estimated Parameters)
  \item Desired output Estimated Input
  \end{itemize}
\end{frame}


\begin{frame}
  \frametitle{PStrick test}
  \begin{figure}[h]
  \begin{center}
  \begin{pspicture}(-5,-2)(5,2)
\psline(-5,-2)(5,2)
\rput(-4, 1){\Rnode{mX}{$\{\bar{\textbf{X}},\textbf{P}_{xx}\}$}}
\rput(-4,-1){\Rnode{Xi}{$\textbf{\Large{X}}_i$}}
\rput( 0, 0){\Rnode{F}{$\textbf{\Large{f}}$}}
\rput( 4,-1){\Rnode{Yi}{$\textbf{\Large{Y}}_i$}}
\rput( 4, 1){\Rnode{mY}{$\{\bar{\textbf{Y}},\textbf{P}_{yy}\}$}}
\psellipticarc{<-}(0,0)(0.5,1){60}{320}\rput(0.5,1){\Large{i}}
%%%%%%%%%%%%%%%%%%%%%%%%%%%%%%%%%%%%%%%%%%%%%%%%%%%%
\ncline{->}{mX}{Xi}\aput{:U}{unscented}
\ncline{->}{Yi}{mY}\aput{:U}{unscented}
\ncline[linestyle=dotted]{->}{Xi}{F}
\ncline[linestyle=dotted]{->}{F}{Yi}
\end{pspicture}
\end{center}
\end{figure}
\end{frame}

\begin{frame}
  \frametitle{Simple slide with three points shown all at once}   % Insert frame title between curly braces
  \begin{itemize}
  \item Point 1
  \item Point 2
  \item Point 3
  \end{itemize}
\end{frame}


\begin{frame}
  \frametitle{Simple slide with three points shown in succession}   % Insert frame title between curly braces

  \begin{itemize}
  \item<1-> Point 1 (Click ``Next Page'' to see Point 2) % Use Next Page to go to Point 2
  \item<2-> Point 2  % Use Next Page to go to Point 3
  \item<3-> Point 3
  \end{itemize}
\end{frame}


\begin{frame}
  \frametitle{Slide with two columns: items and a graphic}   % Insert frame title between curly braces
  \begin{columns}[c]
  \column{2in}  % slides are 3in high by 5in wide
  \begin{itemize}
  \item colum1 item1
  \item colum1 item2
  \end{itemize}
  \column{2in}
  \framebox{Insert graphic here}
  \end{columns}
\end{frame}

\end{document}
