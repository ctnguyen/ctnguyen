%Texlive-full Version 3.141592-1.40.3 (Web2C 7.5.6)
%Kile Version 2.0.83

\documentclass[a4paper,10pt]{article}
\usepackage[utf8x]{inputenc}

\usepackage{lmodern}
\usepackage[a4paper]{geometry}

\usepackage{hyperref}

\usepackage{amsmath}
\usepackage{amssymb}

\usepackage{pstricks}
\usepackage{pst-node}


\begin{document}
%%%%%%%%%%%%%%%%%% DOCUMENT TITLE %%%%%%%%%%%%%%%%%%%%%%%%%
\begin{center}Memento Basis Requirement knowledge for Laure Elie Master\end{center}
%%%%%%%%%%%%%%%%%% DOCUMENT TITLE %%%%%%%%%%%%%%%%%%%%%%%%%
\section{Topology Notions}
\paragraph{Countable set} $K$ is a countable set if either it is empty or there exists a surjection from $\mathbb{N}$ onto $K$ 


\paragraph{Complete metric space} A metric space $\textbf{X}$ is complete if every Cauchy sequence in $\textbf{X}$ converges in $\textbf{X}$ 

\paragraph{Topological separable space} A topological space having a countable dense subset.

\paragraph{Topological countable basis} A subset of the topology that an arbitrary open sets can be written as countable union of this subset

\subsection{Banach space}
\paragraph{Definition} A Banach space is a complete normed vector space, noting ($\mathcal{B}, \|.\|$). Banach space become a Hilbert space if and only if the norm satisfy the parallelogram identity
\[
\| a+b \|^2  + \| a-b \|^2 = 2\|a\|^2  + 2\|b\|^2 
\]

\subsection{Hilbert space} 
\paragraph{Definition} Hilbert space is a Banach space having a inner product $<.,.>$ associating to its norm
\[
\| x \| = \sqrt{<x,x>}
\]





\section{Mesurable space}
\paragraph{$\sigma$-algebra} Giving a set $\textbf{X}$, a $\sigma$-algebra of $\textbf{X}$ is a familly of subsets $\Sigma$ satisfying 
\begin{enumerate}
 \item $\emptyset \in \Sigma$
 \item $\Sigma$ is closed under complement : $ E \in \Sigma \Longrightarrow E^{c} \in \Sigma$
 \item $\Sigma$ is closed under countable union : $ \langle E_{n} \rangle_{n \in \mathbb{N}} \in \Sigma \Longrightarrow \cup E_{n} \in \Sigma $ 
\end{enumerate}
It imply that $\sigma$-algebra is stable by other set operators as difference, symetric difference, finite intersection, union, $\textit{countable}$ intersection, union. 

\paragraph{$\lambda$-systems (Dynkin system)} Giving a set $\textbf{X}$, a $\lambda$-system of $\textbf{X}$ is a familly of subsets $\varLambda$ satisfying 
\begin{enumerate}
 \item $\textbf{X} \in \varLambda$
 \item $\varLambda$ is closed under relative complement : $ A,B \in \varLambda, A \subset B \Longrightarrow B \backslash A \in \varLambda$
 \item $\varLambda$ is closed under increasing countable union : $ \langle A_{n} \rangle_{n \in \mathbb{N}} \in \varLambda, A_n \subset A_{n+1} \Longrightarrow \cup A_{n} \in \varLambda $ 
\end{enumerate}


\paragraph{$\pi$-systems} Giving a set $\textbf{X}$, a $\pi$-system of $\textbf{X}$ is a familly of subsets $\xi$ satisfying 
\begin{enumerate}
 \item $\xi$ is non empty.
 \item $\xi$ is closed under finite intersection.
\end{enumerate}

\paragraph{Ring} Giving a set $\textbf{X}$, a ring of $\textbf{X}$ is a familly of subsets $\textbf{R}$ satisfying 
\begin{enumerate}
 \item $\emptyset \in \textbf{R}$
 \item $\textbf{R}$ is closed under finite union :$ A,B \in \textbf{R} \Longrightarrow A \cup B \in \textbf{R} $
 \item $\textbf{R}$ is closed under relative complement : $ A,B \in \textbf{R} \Longrightarrow B \backslash A \in \textbf{R}$
\end{enumerate}

\paragraph{Transport lemma (generated $\sigma$-algebra by function)} 
\[
\text{Let} \hspace{5mm} 
\left\{ 
\begin{array}{l}
 f : X \longrightarrow Y \\
 \xi \in \mathcal{P}(Y)
\end{array}\right.
\hspace{5mm} \text{then} \hspace{5mm}
\sigma(f^{-1}(\xi)) = f^{-1}(\sigma(\xi))
\]

\subsection{Mesurable function}

\paragraph{Definition}
Let $(X,\mathcal{A})$ and $(Y,\mathcal{B})$ be measurable spaces. A function $f:X \longrightarrow Y$ is said to be measurable iff 
\[
\forall \text{B} \in \mathcal{B} \hbox{ , } f^{-1}(B) \in \mathcal{A} 
\]

\paragraph{Theorem} Let $(X,\mathcal{A})$ and $(Y,\mathcal{B})$ be measurable spaces and $\mathcal{B} = \sigma(\xi)$, then 
\[
f \text{ is measurable } \Longleftrightarrow f^{-1}(\xi) \subset \mathcal{A}
\]
For real functions (in $\mathbb{R}$), every operation like $+,-,*,/,\text{sup},\text{inf},\overline{\text{lim}}, \underline{\text{lim}}$ ... give a measurable function.

\subsection{Mesure Space}
\paragraph{Definition} A Mesure space is a triple $(X,\Sigma,\mu)$ where
\begin{itemize}
 \item $X$ is a set, and $\Sigma$ is the $\sigma$-algebra of $X$
 \item $\mu : \Sigma \longrightarrow \bar{\mathbb{R}}_{+}$ is a positive function such that
 \begin{enumerate}
  \item $\mu(\emptyset) = 0$
  \item if $\langle E_{n} \rangle_{n \in \mathbb{N}}$ is a disjoint sequence set in $\Sigma$, then $\mu(\uplus E_{n}) = \sum_{n \in \mathbb{N}}\mu(E_n) $
 \end{enumerate}
\end{itemize}
Appart general cases, there are several special mesures as point-supported mesure, counting mesure, dirac mesure...

\paragraph{Elementary properties of mesure}
\begin{enumerate}
 \item monotonicity : for all $E,F \in \Sigma$ 
 \begin{itemize}
  \item $E \subseteq F \Longrightarrow \mu(E) \leq \mu(F)$
  \item if $\mu(F) < \infty  \Longrightarrow \mu(E\backslash F) = \mu(E) - \mu(F)$
 \end{itemize}

 \item strong additivity : $\mu(E\cup F) + \mu(E\cap F) = \mu(E) + \mu(F)$ for all $E,F \in \Sigma$ 
 \item left-continuity : if $\langle E_{n} \rangle_{n \in \mathbb{N}}$ a non-decreasing sequence in $\Sigma$ then 
 $\mu(\cup^{\uparrow} E_{n}) = \text{lim}^{\uparrow} \mu(E_{n})$
 \item right-continuity : if $\langle E_{n} \rangle_{n \in \mathbb{N}}$ a non-increasing sequence in $\Sigma$ then 
 $\mu(\cup^{\downarrow} E_{n}) = \text{lim}^{\downarrow} \mu(E_{n})$
 \item subadditivity : for all $\langle E_{n} \rangle_{n \in \mathbb{N}}$ a sequence in $\Sigma$ then $\mu(\cup E_{n}) \leq \sum_{n \in \mathbb{N}}\mu(E_n) $
\end{enumerate}

\paragraph{Negligible sets} $A \subset X$ is a negligible set if $\exists E \in \Sigma, A \subset E, \text{ and } \mu(E)=0$
\paragraph{Conegligible sets} $A \subset X$ is a conegligible set if $X \backslash A$ is negligible.

\paragraph{}There are two fundamental notions in measure theory : the existance and uniqueness of the mesure. 
\begin{itemize}
 \item The uniqueness part is easier, based on the monotone class theorem, state that if two $\sigma$-finite measures coincide on a certain $\pi$-system generating the $\sigma$-field, then these two measures are equal.  
 \item The existance part is build by the outer measure. The idea is by building a pre-measure on a ring, then one can extend to a measure on the $\sigma$-algebra generated by the ring. 
\end{itemize}

\paragraph{Regular measure}
Let $(X, \mathcal{O}(X))$ a topological space, and $\mu$ a measure defined on the Borel algebra $\mathcal{B}(X) = \sigma(\mathcal{O}(X))$. If for all set $A \in \mathcal{B}(X)$ 
\[
\left\|
\begin{array}{l}
 \mu(A) = \text{inf}\{\mu(O) | A \subset O, O \text{ open set}\} \text{ then } \mu \text{ is outer regular } \\
 \mu(A) = \text{sup}\{\mu(K) | K \subset A, K \text{ compact set}\} \text{ then } \mu \text{ is inner regular } 
\end{array}
\right.
\]
A theorem state that all is measure (finite on compacts) on a metric space, locally compact and separable is regular. Especially the Lebesgue measure on $\mathbb{R}^d$, or all measure defined by a density $\mu=f.\lambda_d$ where $f$ is locally integrable (i.e $\int_K fd\lambda_d < \infty \hbox{ , } K \text{compact}$) are all regular.

\section{Integration}
\paragraph{Simple function} is a finite linear combination of indicator functions on a measurable sets. 
\[
\varphi(\omega) = \sum_{k=1}^{n} a_k \mathbf{1}_{A_k}(\omega)
\]
\begin{itemize}
 \item Any non-negative mesurable function is the pointwise limit of a monotonic increasing sequence of non-negative simple functions. 
 \item Any bounded mesurable function is the uniform pointwise limit of a sequence of simple functions.
\end{itemize}
\paragraph{Lebesgue integration } for a non-negative measurable function $f$ :
\[
\int fd\mu = \text{sup} \{ \int \varphi d\mu \hbox{ , } \varphi \leq f \hbox{ , } \varphi \text{ non-negative simple function} \}
\]
This definition give the monotonicity, the additivity and the positive homogeneity of the Lebesgue integration, propreties deduced directly from the simple functions.

\paragraph{Monotone convergences} Let $f_n$ be a sequence of measurable functions, non-negative, non-decreasing ($0 \leq f_n \leq f_{n+1}$), then
\[
\int \underset{n}{\text{lim}} f_n d\mu = \underset{n}{\text{lim}} \int  f_n d\mu 
\]

\paragraph{Fatou's lemma}
Let $f_n$ be a sequence of $\textbf{non-negative}$ measurable functions, then
\[
\int \underset{n}{\varliminf} f_n d\mu \leq \underset{n}{\varliminf} \int  f_n d\mu 
\]

\paragraph{Dominated convergences}
Let $f_n$ a sequence of measurable functions, satisfying 
\begin{itemize}
 \item $f_n \longrightarrow f $ pointwise almost everywhere
 \item $f_n \leq g \hbox{ , } \forall n$ where $g$ is integrable 
\end{itemize}
then
\[
\int \underset{n}{\text{lim}} f_n d\mu = \underset{n}{\text{lim}} \int  f_n d\mu 
\hspace{1cm}
\text{or equivalently}
\hspace{1cm}
 \underset{n}{\text{lim}} \int |f_n -f| d\mu =0
\]

\subsection{\texorpdfstring{$\textbf{L}^p$}{LpSpace} space}
\paragraph{Holder inequality} Let $p,p \in [1,\infty]$ which are Holder conjugates $\frac{1}{p} + \frac{1}{p} = 1$, then
\[
\| f.g \|_{1} \leq \| f \|_{p}  \| g \|_{q}  
\]
If $p,p \in (1,\infty)$ and $ \| f \|_{p} + \| g \|_{q} <  \infty$ then there are equality iff $\alpha |f|^p = \beta |g|^q$ $\mu$-a.e

\paragraph{Minkowski inequality}

\paragraph{Densities in $\textbf{L}^p$ spaces with $1 \leq p < \infty$}

\paragraph{$\textbf{L}^{\infty}$ space}
common point with others p, particular point of p=infty

























\section{Functional analysis}
\paragraph{Banach space}
\paragraph{Hilbert space}


\paragraph{Banach fixed-point theorem}
Let $(X,d)$ be a non-empty complete metric space with a contraction mapping $T:X\longrightarrow X$. Then $T$ admits a unique fixed-point $x^{*}$ in $X$ satisfying $T(x^{*}) = x^{*}$. Furthermore, $x^{*}$ can be found as followsm star with an arbitrary element $x_0$ in $X$ and define a sequence ${x_n}$ by $x_{n+1}=T(x_n)$, then $x_n \longrightarrow x^*$

\paragraph{Hilbert projection theorem}


\paragraph{Implicit function theorem}

\paragraph{Lax–Milgram theorem} Let $b$ be a bounded coercive bilinear functional on a Hilbert space $H$. Then for every bounded linear functional $L$ on $H$ there exists a unique $u_{L} \in H$ such that 
\[
L(v) = b(v,u_{L}) \hspace{2cm} ,\forall v \in H
\]

\paragraph{Riesz representation theorem}
\paragraph{Riesz–Markov–Kakutani representation theorem}
\paragraph{densities in functional spaces}

\section{Differential equations}

\end{document}