%Texlive-full Version 3.141592-1.40.3 (Web2C 7.5.6)
%Kile Version 2.0.83

\documentclass[a4paper,10pt]{article}
\usepackage[utf8x]{inputenc}
\usepackage{graphicx}

\usepackage{lmodern}
\usepackage[a4paper]{geometry}

\usepackage{hyperref}

\usepackage{amsmath}
\usepackage{amssymb}

\usepackage{pstricks}
\usepackage{pst-node}


\begin{document}
%%%%%%%%%%%%%%%%%% DOCUMENT TITLE %%%%%%%%%%%%%%%%%%%%%%%%%
\begin{center}\resizebox{9cm}{0.6cm}{Quant's interview 1 - Math Derivatives}\end{center}
%%%%%%%%%%%%%%%%%% DOCUMENT TITLE %%%%%%%%%%%%%%%%%%%%%%%%%
These formula taken from books ~\cite{mathfiElliot} and ~\cite{mathfiMarek} 
\section{Basis financial Option}
\subsection{European Option}
- Call put parity
- Bound for European Call
- Bound for European Put
- Bound for American Call, Put
\section{Black Scholes Model}
There are two approach leads to the Black-Scholes tools
\begin{itemize}
 \item Black-Scholes formula : Starting from the self-financing porfolio, based on the probabiliste approach (martingale theory), one can build the closed formula of price for every attainable contingent claim. 
 \item Black-Scholes PDE : Starting from the riskless porfolio (delta neutral) idea, give the Black-Scholes PDE.
\end{itemize}
Lecturer can see detail formulations from ~\cite{mathfiElliot}.ch7 , ~\cite{mathfiMarek}.ch3 or ~\cite{rouah}.  
\paragraph{The Model} Black-Scholes define the dynamic of risky and riskless assets :
\[
\frac{dS^1_t}{S^1_t} = \mu dt + \sigma d W_t   \hspace{2cm} \frac{dS^0_t}{S^0_t} = r dt
\]
\paragraph{The Self-financing porfolio} or self-financing porfolio $\phi = (H^0,H^1)$ is defined by
\[
V_t(\phi) = H^0_t S^0_t + H^1_t S^1_t   
\hspace{1.5cm} 
dV_t = H^0_t dS^0_t + H^1_t dS^1_t 
\hspace{1cm} \longrightarrow
d\widetilde{V}_t = H^1_t d\widetilde{S}^1_t 
\]
\paragraph{The Girsanov Probability transform}
\[
\theta = \frac{\mu - r}{\sigma} 
\hspace{1cm} 
\frac{d\mathbb{P}^*}{d\mathbb{P}}|_{\mathcal{F}_t} = \text{exp}\{ -\int_0^t \theta dW_s - \frac{1}{2} \int_0^t \theta^2 ds \}
\hspace{1cm} 
W^*_t = W_t + \int_0^t \theta ds
\]
\paragraph{Comparing in two probabilities}
\begin{center}
\begin{tabular}{l|l}
 Under $\mathbb{P}$ & Under $\mathbb{P}^*$                                                      \\[6pt]
 $ dS^1_t = S^1_t( \mu dt + \sigma d W_t) $       &  $ dS^1_t = S^1_t( r dt + \sigma d W^*_t) $ \\[3pt]
 $ dV_t = H^0_t dS^0_t + H^1_t dS^1_t     $       &  $ dV_t = r V_t dt + \sigma H^1_t dW^*_t  $ \\[3pt]
 $ d\widetilde{S}^1_t = \widetilde{S}^1_t( (\mu-r)dt + \sigma dW_t ) $ & $d\widetilde{S}^1_t = \widetilde{S}^1_t \sigma dW^*_t  $ \\[3pt]
 $ d\widetilde{V}_t = H^1_t d\widetilde{S}^1_t $ & $d\widetilde{V}_t = H^1_t d\widetilde{S}^1_t =  \sigma H^1_t \widetilde{S}^1_t dW^*_t $
\end{tabular}
\end{center}


- Black Scholes Formula for European Call, Put
- Black-Scholes Equations
- Forward price formula
- Numeraire Change formula
\subsection{Exotic Options}
- every Exotic Option, see Hull, Martingale Finance, ...
- packages options, 
- forward-start options, 
- chooser options, 
- compound options, 
- digital options, 
- barrier options, 
- look back options, 
- asian options, 
- basket options, 
- quantile options, 
- others
\section{Greek letters}
- every greek and its graphics
- relation between greeks
- greek letter in black-scholes formula
\section{Volatility}
\section{Interest rate}
\section{Models}
- see Martingale Method In Financial Modelling books
\subsection{Bachelier Model}
\subsection{Black Model}
\subsection{Vasicek Model}
\subsection{Heston Model}
\subsection{Hull-White Model}
\subsection{Heath-Jarrow-Morton Model}
\subsection{Cox-Ingersoll-Ross model}
\subsection{Constant Elasticity of Variance Model}
\section{Useful mathematical formula}
- Ito multidim
- Girsanov multidim
- Faeymann Kac
- Martingale Representation theorem
- 
\section{Lagrange Multiplier}
\section{Statistical Linear Gaussian Model}
\bibliographystyle{siam}
\bibliography{QuantBib}
\end{document}
