%Texlive-full Version 3.141592-1.40.3 (Web2C 7.5.6)
%Kile Version 2.0.83

\documentclass[a4paper,10pt]{article}
\usepackage[utf8x]{inputenc}
\usepackage{graphicx}

\usepackage{lmodern}
\usepackage[a4paper]{geometry}

\usepackage{hyperref}

\usepackage{amsmath}
\usepackage{amssymb}

\usepackage{pstricks}
\usepackage{pst-node}

\usepackage{booktabs}
\begin{document}
%%%%%%%%%%%%%%%%%% DOCUMENT TITLE %%%%%%%%%%%%%%%%%%%%%%%%%
\begin{center}\resizebox{9cm}{0.6cm}{Quant's interview 1 - Math Derivatives}\end{center}
%%%%%%%%%%%%%%%%%% DOCUMENT TITLE %%%%%%%%%%%%%%%%%%%%%%%%%
\section{Some trivial probabilities}

\subsection{Common}

\subsection{Binomial, multinomial, random warks}
Further remaining the formulas, one has to reformulate the urn models and understand the signification of variables
\paragraph{Binomial} $X\sim \mathcal{B}(n,\theta)$ when,
\[
X \in \{0,1 .. , n\} , 
\hspace{10mm}
P(X=k)=C^k_n\theta ^k (1-\theta)^{n-k} 
\hspace{10mm}
\left\{
\begin{array}{l}
\mathbb{E}[X]=n\theta \\
\textbf{Var}[X]=n \theta (1-\theta) 
\end{array}\right.
\]
\paragraph{Multinomial}  $ \textbf{X} \sim \mathcal{M}(n,\textbf{p} ) \hbox{ , } \textbf{p}\in \mathbb{R}^k $ when \footnote{Use of multi-index notation : $ \hspace{1cm} |\textbf{x}|=x_1 + ... + x_k \hspace{1cm} \textbf{x}!=x_1! ... x_k!  \hspace{1cm} \textbf{p}^{\textbf{x}} = p_1^{x_1}...p_k^{x_k}  \hspace{1cm} $, multinomial is a generalization of the binomial.},
\[
\textbf{X} \in \Omega = \{ \textbf{x} \in \mathbb{N}^k, |\textbf{x}| = n \} , 
\hspace{10mm}
P(\textbf{X}=\textbf{x})= \frac{n!}{\textbf{x}!}\textbf{p}^{\textbf{x}}
\hspace{10mm}
\left\{
\begin{array}{l}
\mathbb{E}[X_i]=np_i \\
\textbf{Var}[X_i]=n p_i (1-p_i) 
\end{array}\right.
\]
\paragraph{Random walk and Gambler's ruin problem} is a markov based problem modeled as random walks ~\cite{KARL}. A Gambler play a coin-flipping game, he initiate with $i\$$ and stop the game if his fortune reach $W\$$ (win) or $0\$$ (lost). Each flip of coin he gains/looses $1\$$ with probability $p \text{ or } q$. Note $P_i^{W}$ the probability he win the game, then
\[
P_i^{W} = \frac{1- (\frac{q}{p})^i }{1- (\frac{q}{p})^W } \hspace{0.5cm} \text{  if } p\neq q
\hspace{3cm}
P_i^{W} = \frac{i}{W} \hspace{0.5cm}\text{  if } p = q = \frac{1}{2}
\]
See also St-Peterburg paradox.
\subsection{Gaussians}
\paragraph{Normal variable} $X \sim \mathcal{N}(\mu,\sigma^2)$ when $X \in \mathbb{R}$ and,
\[
f_{X}(x) = \frac{1}{\sigma \sqrt{2\pi}} exp (-\frac{(x-\mu)^2}{2\sigma^2})
\hspace{5mm}
\left\{
\begin{array}{l}
\mathbb{E}[X]=\mu \\
\textbf{Var}[X]=\sigma^2
\end{array}\right.
\hspace{5mm}
\left\{
\begin{array}{l}
\mathcal{L}_{X}(s) =  \text{exp}(\mu s  + \frac{1}{2} \sigma^2 s^2) \hbox{ , } s\in\mathbb{C} \\
\varphi_{X}(t)     =  \text{exp}(i\mu t - \frac{1}{2} \sigma^2 t^2) \hbox{ , } t\in\mathbb{R}
\end{array}\right.
\]
\[
\frac{X - \mu}{\sigma} \sim \mathcal{N}(0,1)
\hspace{10mm}
f_{X}(x) = \frac{1}{\sigma}f_{\mathcal{N}(0,1)}(\frac{x-\mu}{\sigma})
\hspace{10mm}
\mathbb{F}_{X}(x) = \frac{1}{\sigma}\phi(\frac{x-\mu}{\sigma})
\]
\paragraph{Central Limit Theorem}
\[
\text{if} \hspace{5mm} (X_{n})_{(n\geqslant 0)} \in \textbf{L}^2 \text{  i.i.d,} \hspace{5mm} \text{ then  } \hspace{5mm}
\frac{\overline{X}_n - \mu}{\frac{\sigma}{\sqrt{n}}} \rightsquigarrow   \mathcal{N}(0,1)
 \hspace{5mm}  \text{    or} \hspace{5mm}  
\sqrt{n}(\overline{X}_n - \mu) \rightsquigarrow   \mathcal{N}(0,\sigma^{2})
\]
\subsection{Brownian motion} 
\paragraph{Definition} Brownian motion is a \textit{continuous} procces $X$ satisfying one of following equivalent properties  (see~\cite{MONIQUE}sec1.4):
\begin{itemize}
 \item $X$ has stationary independent increments, and for any $t>0$, $X_t \sim \mathcal{N}(0,t)$
 \item $X$ is a Gaussian process, with mean 0 and covariance $t\wedge s$
 \item $X_t$ and $(X_t^2 - t)$ are local martingales.
 \item $\forall \lambda \hspace{0.5cm} \text{exp}\{  \lambda X - \frac{\lambda^2}{2}t \}_{t\geq0}$ is a local martingale
 \item $\forall \lambda \hspace{0.5cm} \text{exp}\{ i\lambda X + \frac{\lambda^2}{2}t \}_{t\geq0}$ is a local martingale
\end{itemize}
\paragraph{Properties}
If $B_t$ is a brownian motion, then \footnote{$\tau$ is a a.s finite stopping time} (see~\cite{PETER})
\begin{itemize}
 \item $\hspace{2cm} (B_t-B_{\tau})_{t\geq \tau} \hspace{1cm}  (\frac{1}{a}B_{a^2 t}) \hspace{1cm}  (tB_{\frac{1}{t}})$ are brownian motion.
 \item a.s, $\hspace{2cm} \text{Lim}\frac{B_t}{t} = 0   \hspace{1cm} \varlimsup\frac{B_n}{\sqrt{n}}=+\infty \hspace{1cm} \varliminf\frac{B_n}{\sqrt{n}}=-\infty$
 \item a.s, Bronian motion is nowhere differentiable and on any small interval $t\in[a,b]$, brownian motion is non monotone.
 \item \textbf{Reflexion principle} : $\tilde{B}_t = \mathbf{1}_{\{t\leq \tau\}} B_t + \mathbf{1}_{\{t>\tau\}} (2B_T - B_t) $ is Brownian motion  
\end{itemize}
\paragraph{Multidimensional}
If $\textbf{B}_t$ is a $d$-Brownian motion, then (see~\cite{OKSENDAL}8.4.2)
\[
Y_t = \int_0^t v^t d\textbf{B}_t \text{ is a reel Brownian} 
\hspace{1.5cm} \Longleftrightarrow \hspace{1.5cm}
vv^t = \textbf{I}_d
\]
\paragraph{Correlated Bronian motion } $\textbf{X}_t$ is similarily defined as a standard Brownia, i.e a continuous process having stationary independent increments and $\textbf{X}_t - \textbf{X}_s \sim \mathcal{N}(0,(t-s).\rho)$ where $\rho$ is the correlation matrix\footnote{Note that correlation matrix is different from covariance matrix. Indeed it is a 'normalized covariance'. For being a correlation matrix, $\rho$ has to satisfied somes conditions, $\rho_{ii}=1$ and $\rho_{ij} \in ]-1,1[$ for exemple}. It is equivalent to say that $\textbf{X}_t = \rho^{\frac{1}{2}} \textbf{B}_t$ where $\rho^{\frac{1}{2}}$ is the Cholesky decomposition of $\rho$. Note that $\rho^{\frac{1}{2}}$ is a lower triangular matrix (see~\cite{MAREK}7.1.9 for exemple in two dimensinal).

\subsection{Stochastic calculus}
\paragraph{Wiener Integral} is $\int_a^t f(s)dB_s$ for all $f\in \mathbb{L}^2([a,b])$ which is a $\mathbb{L}^2$-martingale such that
\[
\mathbb{E}[\int_a^t f(s)dB_s \int_a^t g(s)dB_s] = \int_a^t f(s)g(s)ds
\hspace{1.5cm}
\int_a^t f(s)dB_s \sim \mathcal{N}(0, \int_a^t f(s)^2ds)
\]
\paragraph{Ito Integral} is $\int_a^t f(s,w)dB_s$ for all $f\in \mathbb{L}^2([a,b],\Omega)$ which is a $\mathbb{L}^2$-martingale a.s continuous of zero mean and
\[
\mathbb{E}[\int_a^t f(s,w)dB_s \int_a^t g(s,w)dB_s] = \int_a^t \mathbb{E}[f(s)g(s)]ds
\]
If $f\in \mathbb{L}_{\text{loc}}^2([a,b],\Omega)$ then $\int_a^t f(s,w)dB_s$ is a local martingale and the isometry property is not remaining.
\paragraph{Ito Formula}
Let $\textbf{X}_t$ be a multidimensional Ito proccess
\[
d\textbf{X}_t = u(t,w)dt + v(t,w)d\textbf{B}_t
\]
and $f(t,\textbf{X})$ be a $\mathbb{C}^{1,2}$ function, then
\[
df = \partial_t f dt + \triangledown f \cdot d\textbf{X} + \frac{1}{2} \text{Tr}(vv^T \textbf{D}^2f  )
\]
\paragraph{Ito presentation theorem}
if $F \in \mathbb{L}^2(\mathcal{F}_T)$, there existe a unique stochastic process $f\in \mathbb{L}^2([0,T],\Omega)$ such that
\[
F = \mathbb{E}[F] + \int_0^T f(t,w)dB_t
\]
\paragraph{Martingale presentation theorem}
If $M_t$ is a $\mathbb{L}^2$-martingale, then for all $t$ there existe a unique stochastic process $g\in \mathbb{L}^2([0,t],\Omega)$ such that
\[
M_t = \mathbb{E}[M_0] + \int_0^t g(s,w)dB_s
\]
\paragraph{Girsanov theorem}\footnote{the theorem can apply identically for d dimension}
\[
\text{Let }\hspace{1cm}
\textbf{h}\in \mathbb{L}_{\text{loc}}^2([0,T],\Omega)
\hspace{2cm}
\xi_h(t) = \text{exp}\{ \int_0^t \textbf{h}_s d\textbf{B}_s - \frac{1}{2} \int_0^t |\textbf{h}|_s^2 ds \}
\]
Under the Novikov\footnote{$\mathbb{E}\text{exp}(\frac{1}{2}\int_0^T \textbf{h}^2_sdt) < \infty $} condition, then $\xi_h(t)$ is a $\mathbb{P}$-martingale with $\mathbb{E}[\xi_h(t)]=1$ and
\[
\textbf{W}_t = \textbf{B}_t - \int_0^t \textbf{h}_sds \text{ is a Brownien Motion under }d\mathbb{Q} =\xi_h(T) d\mathbb{P}
\]
\paragraph{Girsanov transform for correlated Brownian} is the way to translate the correlated Brownian. When having a correlated Brownian, how to translate it through the Girsanov theorem?
\begin{center}
\begin{pspicture}(-7,-1.3)(7,1.3) 
%\psline(-7,-1.3)(7,1.3) 
\rput( -4, 0)
{
	\begin{tabular}{l}
	  $\textbf{W}_t$ d-Brownian under $\mathbb{P}$ \\
	  $\textbf{W}^*_t$ translated d-Brownian under $\mathbb{P}^*$ \\
	  $\widetilde{\textbf{W}}_t$ Correlated d-Brownian  under $\mathbb{P}$ \\
	  $\hat{\textbf{W}}_t$ Correlated d-Brownian under $\mathbb{P}^*$
	\end{tabular}
}
\rput( 0, 0){\Rnode{w1}{$\textbf{W}_t$}}
\rput( 3, 1){\Rnode{w2}{$\textbf{W}_t^*$}}
\rput( 3, -1){\Rnode{w3}{$\widetilde{\textbf{W}}_t$}}
\rput( 6, 0){\Rnode{w4}{$\hat{\textbf{W}}_t$}}
\ncline{->}{w1}{w3}\ncput*{1}
\ncline{->}{w1}{w2}\ncput*{2}
\ncline{->}{w2}{w4}\ncput*{3}
\ncline{->}{w3}{w4}\ncput*{4}
\end{pspicture}
\end{center}
By order as below, from the correlated Brownian $\widetilde{\textbf{W}}_t$ with correlation matrix $\rho$, which is indeed $\rho^{\frac{1}{2}}\textbf{W}_t $ ( we find $\rho^{\frac{1}{2}}$ by Cholesky ). At the same time, $\textbf{W}^*_t$ is a Brownian under $\mathbb{P}^*$ which is translated from $\textbf{W}_t$ through $\textbf{h}_t$. Finally we find $\hat{\textbf{W}}_t$ under $\mathbb{P}^*$ the Brownian translated of $\widetilde{\textbf{W}}_t$ through  $\textbf{h}_t$.
\begin{enumerate}
 \item $\textbf{d}\widetilde{\textbf{W}}_t = \rho^{\frac{1}{2}}\textbf{dW}_t $
 \item $\textbf{dW}^*_t=\textbf{dW}_t - \textbf{h}_tdt$ 
 \item $\textbf{d}\hat{\textbf{W}}_t=\rho^{\frac{1}{2}}\textbf{dW}^*_t = \rho^{\frac{1}{2}}\textbf{dW}_t - \rho^{\frac{1}{2}}\textbf{h}_tdt$
 \item $\textbf{d}\hat{\textbf{W}}_t=\textbf{d}\widetilde{\textbf{W}}_t - \rho^{\frac{1}{2}}\textbf{h}_tdt$
\end{enumerate}


\paragraph{Geometric Brownian Motion}
\[
dx_t = x_t (\mu dt + \sigma dB_t ) 
\hspace{1cm} \longrightarrow \hspace{1cm}
x_t = x_0\text{exp}\{(\mu-\frac{\sigma^2}{2})t +\sigma B_t  \}
\]
\paragraph{Ornstein–Uhlenbeck process}
\[
dx_t = \theta (\mu-x_t) dt + \sigma dB_t
\hspace{1cm} \longrightarrow \hspace{1cm}
x_t = x_0e^{-\theta t} + \mu(1-e^{-\theta t}) +\int_0^t \sigma e^{\theta(s-t)}dB_s
\]
where $\theta,\mu,\sigma >0$





\section{Basis financial Option}
In this section\footnote{ All relations in this section are model-free. That mean they can be demonstrated without assumption of a model, only by the non arbitrage theory} $B_{(t,T)}$ and $F_{(t,T)}$ are zero-coupon price and T-Forward price of the stock $S_t$ in question, see  ~\cite{HULL} 
\subsection{European Option}
\paragraph{Call-Put parity} 
\[
C_t - P_t = S_t - B(t,T).K
\]
\paragraph{Bound for European options}
\[
B_{(t,T)}(K-F_{(t,T)})^+ \leq P_t \leq B_{(t,T)}K 
\hspace{1cm}
B_{(t,T)}(F_{(t,T)}-K)^+ \leq C_t \leq B_{(t,T)}F_{(t,T)}
\]
\paragraph{Bound for American options}
\[
S_t - K \leq C^A_t - P^A_t \leq S_t - B(t,T)K
\hspace{1cm}
C^A_t \geq (S_t - K)^+
\hspace{1cm}
P^A_t \geq (K - S_t)^+
\]


\paragraph{Three no-arbitrage inequalities}
\begin{itemize}
 \item Call Spread : $K_1 < K_2 \rightarrow C(K_1)-C(K_2) \geq 0 \rightarrow \frac{\partial C }{\partial K} \leq 0  $
 \item Butterfly Spread : $C(K-\epsilon) - 2C(K) + C(K+\epsilon) \geq 0 \rightarrow \frac{\partial^2 C }{\partial K^2} \geq 0 $
 \item Calandar Spread : $T_1<T_2  \rightarrow C_{t}(T_1,K) \leq C_{t}(T_2,e^{r(T_2-T_1)} K) \rightarrow \frac{\partial C }{\partial T} \geq 0 \text{ if } r=0 $
\end{itemize}
- Bound for American Call, Put
\subsection{Trading strategies}
Every strategies in Hull,
\paragraph{Call,Put calandar}
\paragraph{Alpha hedging}
\paragraph{Delta hedging} consiste in holding $\Delta$ quantity of the underlying asset in order to neutralize the asset price move when having a option. Assumed a investor short one option $V$ and hold $\Delta$ asset. 
\[
\varPi(t,S_t) = -V(t,S_t) + \Delta.S_t
\hspace{2cm}
\text{ Delta Hedging }
\Delta = \frac{\partial V}{\partial S}
\]
\paragraph{Delta-Vega hedging} consiste in holding $\Delta$ quantity of underlying asset and $\Delta_{1}$ quantity of another option (whose values depend to volatility\footnote{ Note that every option and stock are about only one underlying asset} ) in order to neutralize the asset price move and the volatility move (see~\cite{GATHERAL}p.4-5). Assumed a investor short one option $V$, hold $\Delta$ asset and $\Delta_{1}$ quantity of an other option $V_1$
\[
\varPi(t,S_t,\sigma_t) = -V(t,S_t,\sigma_t) + \Delta.S_t + \Delta_1 V_1(t,S_t,\sigma_t)
\]
Then Delta-Vega hedging strategy is 
\[
\Delta_1 =\frac{\frac{\partial V}{\partial \sigma} }{\frac{\partial V_1}{\partial \sigma} }
\hspace{2cm}
\Delta   =\frac{\partial V}{\partial S} - \Delta_1 \frac{\partial V_1}{\partial S} 
\]

\section{Black Scholes Model}
There are two approach leads to the Black-Scholes tools
\begin{itemize}
 \item Black-Scholes formula : Starting from the self-financing porfolio, based on the probabiliste approach (martingale theory), one can build the closed formula of price for every attainable contingent claim. 
 \item Black-Scholes PDE : Starting from the riskless porfolio (delta neutral) idea, give the Black-Scholes PDE.
\end{itemize}
Lecturer can see detail formulations from ~\cite{ELLIOT}.ch7 , ~\cite{MAREK}.ch3 or ~\cite{rouah}.  
\paragraph{The Model} Black-Scholes define the dynamic of risky and riskless assets \footnote{In this section, the interest rate and volatility subscripted $t$ mean $r_t$ and $\sigma_t$ is a deterministic function in time} :
\[
\frac{dS^1_t}{S^1_t} = \mu dt + \sigma_t d W_t   \hspace{2cm} \frac{dS^0_t}{S^0_t} = r_t dt
\]
\paragraph{The Self-financing porfolio} or self-financing porfolio $\phi = (H^0,H^1)$ is defined by
\[
V_t(\phi) = H^0_t S^0_t + H^1_t S^1_t   
\hspace{1.5cm} 
dV_t = H^0_t dS^0_t + H^1_t dS^1_t 
\hspace{1cm} \longrightarrow
d\widetilde{V}_t = H^1_t d\widetilde{S}^1_t 
\]
\paragraph{The Girsanov Probability transform}
\[
\theta_t = \frac{\mu - r_t}{\sigma_t} 
\hspace{1cm} 
\frac{d\mathbb{P}^*}{d\mathbb{P}}|_{\mathcal{F}_t} = \text{exp}\{ -\int_0^t \theta_s dW_s - \frac{1}{2} \int_0^t \theta^2_s ds \}
\hspace{1cm} 
W^*_t = W_t + \int_0^t \theta_s ds
\]
\paragraph{Comparing processes's dynamic in two probabilities}
\begin{center}
\begin{tabular}{l|l}
 Under $\mathbb{P}$ & Under $\mathbb{P}^*$                                                      \\[6pt]
 $ dS^1_t = S^1_t( \mu dt + \sigma_t d W_t) $       &  $ dS^1_t = S^1_t( r_t dt + \sigma_t d W^*_t) $ \\[3pt]
 $ dV_t = H^0_t dS^0_t + H^1_t dS^1_t     $       &  $ dV_t = r_t V_t dt + \sigma_t H^1_t dW^*_t  $ \\[3pt]
 $ d\widetilde{S}^1_t = \widetilde{S}^1_t( (\mu-r_t)dt + \sigma_t dW_t ) $ & $d\widetilde{S}^1_t = \widetilde{S}^1_t \sigma_t dW^*_t  $ \\[3pt]
 $ d\widetilde{V}_t = H^1_t d\widetilde{S}^1_t $ & $d\widetilde{V}_t = H^1_t d\widetilde{S}^1_t =  \sigma_t H^1_t \widetilde{S}^1_t dW^*_t $
\end{tabular}
\end{center}
The result under risk-neutral probability is
\[
\begin{array}{l}
\pi_t(h_{T}) = S^0_t \mathbb{E}^*[\frac{h_T}{S^0_T} | \mathcal{F}_t]
\hspace{3cm}
\widetilde{S}^1_t = S^1_0\text{exp}\{ \sigma W^*_t -\frac{\sigma^2}{2}t\}  \\ \\
S^1_t = S^1_0\text{exp}\{  \sigma W^*_t + (r-\frac{\sigma^2}{2})t \} = S^1_s\text{exp}\{  \sigma (W^*_t - W^*_s) + (r-\frac{\sigma^2}{2})(t-s) \}
\end{array}
\]

\paragraph{The prices formula when parameters are constants}
\[
C_t = S^1_t \Phi(d_+) - e^{-r(T-t)} K \Phi(d_-)
\hspace{1cm}
P_t = e^{-r(T-t)} K \Phi(-d_-) - S^1_t \Phi(-d_+)
\]

\[
d\pm = \frac{ \text{ln}(\frac{S^1_t}{K}) + (T-t)( r\pm \frac{\sigma^2}{2})  }{\sigma \sqrt{T-t}} 
\hspace{2cm}
d_- = d_+  - \sigma \sqrt{T-t}
\]
\paragraph{The prices formula when parameters are functions in time}
\[
C_t = S^1_t \Phi(d_+) - e^{- \int^T_t r_udu } K \Phi(d_-)
\hspace{1cm}
P_t = e^{- \int^T_t r_udu } K \Phi(-d_-) - S^1_t \Phi(-d_+)
\]

\[
\overline{\sigma}^2_{t,T} = \int^T_t \sigma^2_u du
\hspace{2cm}
d\pm =  \frac{1}{\overline{\sigma}_{t,T}} \text{ln}(\frac{S^1_t}{e^{- \int^T_t r_udu }.K}) \pm \frac{1}{2}\overline{\sigma}_{t,T}
\hspace{2cm}
d_- = d_+  - \overline{\sigma}_{t,T}
\]
\paragraph{Numeraraire change }
A proccess $Z_t > 0$ can be considered as a numeraire. 
\[
\frac{d\mathbb{P}^{Z}}{d\mathbb{P}^*} | \mathcal{F}_t = \frac{\widetilde{Z}_t }{Z_0} = \frac{1}{Z_0} \frac{Z_t}{S^0_t} 
\]
Then for every $X_t$ asset price process, $\frac{X_t}{Z_t}$ is a $\mathbb{P}^Z$-martingale, as $\frac{X_t}{S^0_t}$ is a $\mathbb{P}^*$-martingale. 
\textbf{T-Forward measure } is the measure using the numeaire zero-coupon $B(t,T)$, under which the T-forward price procces is a martingale.

\paragraph{PDE Black-Scholes}
\[
\frac{\partial C}{\partial t} +
(r_t - q_t)S_t \frac{\partial C}{\partial S} + 
\frac{1}{2}\sigma^2_tS^2_t \frac{\partial^2 C}{\partial S^2} -
r_t C_t =0
\hspace{2cm}
\text{ divident : } dQ_t = S_t q_t dt 
\]
\subsection{Exotic Options}
\paragraph{Packages options} is a combination of standard options, cash and underlying asset.
\begin{description}
 \item[Collars : ] $\textbf{CL}_T = \text{min}\{ \text{max}(S_T,K_1),K_2\} = K_1 +(S_T-K_1)^+ - (S_T - K_2)^+ $ where $K_2>K_1>0$
 \item[Break Forward : ] $\textbf{BF}_T = \text{max}(S_T,F_S^T)-K = (S_T- F_S^T)^+ +F_S^T -K $ where $F_S^T$ is the $T$-forward price of the underlying asset.
 \item[Range Forward : ] $\textbf{RF}_T = \text{max}\{ \text{min}(S_T,K_2),K_1\} - F_S^T  = S_T - F_S^T +(K_1 - S_T)^+ - (S_T-K_2)^+ $ where $K_1<F_S^T<K_2$
\end{description}
\paragraph{Forward-start options} $\textbf{FS}_T = (S_T - KS_{T_0})$ with $K>0$ and $0<T_0<T$


- every Exotic Option, see Hull, Martingale Finance, ...
 
- forward-start options, 
- chooser options, 
- compound options, 
- digital options, 
- barrier options, 
- look back options, 
- asian options, 
- basket options, 
- quantile options, 
- others

\section{Greek letters}
Greek are coefficients studying the sensitivities of \textit{Standard European Option} in relation to other variables (see~\cite{MAREK} s3.1.11)
\paragraph{Delta} $\frac{\partial}{\partial S}$
\[
\Delta_C(t) = \Phi(d_+(t))>0
\hspace{1cm}
\Delta_P(t) = -\Phi(-d_+(t))<0
\hspace{1cm}
\Delta_C(t) - \Delta_P(t) = 1
\]
\paragraph{Gamma} $\frac{\partial^2}{\partial S^2}$
\[
\Gamma_C(t) = \Gamma_P(t) = \frac{\phi(d_+(t)) }{S_t \sigma \sqrt{T-t}} >0
\]
\paragraph{Theta} $\frac{\partial}{\partial T}$
\[
\left\| 
\begin{array}{l}
\Theta_C = \frac{-S_t \phi(d_+(t)) \sigma}{2\sqrt{T-t}} - rKe^{-r(T-t)} \Phi(d_-(t))  <0 \\
\Theta_P = \frac{-S_t \phi(d_+(t)) \sigma}{2\sqrt{T-t}} + rKe^{-r(T-t)} \Phi(-d_-(t)) 
\end{array}\right. 
\hspace{2cm}
\left\{ 
\begin{array}{l}
\phi \text{ normal density} \\
\Theta_P - \Theta_C = rKe^{-r(T-t)}
\end{array}\right. 
\]
\paragraph{Delta - Theta - Gamma of delta-neutral porfolio $\Pi$}
\[
d\Pi = \Theta dt + \frac{1}{2} \Gamma dS^2
\hspace{2cm}
\Theta + \frac{1}{2}\sigma^2 S^2 \Gamma = r\Pi
\]
\paragraph{Vega}$\frac{\partial}{\partial \sigma}$
\[
\mathcal{V}_C = \mathcal{V}_P = S_t \sqrt{T-t} \phi(d_+(t))>0
\]
\paragraph{Rho} $\frac{\partial}{\partial r}$
\[
\left\| 
\begin{array}{l}
\rho_C = (T-t)Ke^{-r(T-t)}\Phi(d_-(t))    > 0 \\
\rho_P = -(T-t)Ke^{-r(T-t)}\Phi(-d_-(t))  < 0
\end{array}\right. 
\]
\paragraph{others}
\[
\partial_{K}C = -e^{-r(T-t)} \Phi(d_-(t)) <0
\hspace{2cm}
\partial_{K}P = e^{-r(T-t)} (1-\Phi(d_-(t))) >0
\]

- every greek and its graphics
- relation between greeks
- greek letter in black-scholes formula
\section{Volatility}
Correlated brownian, implied vol, local vol, sto vol.
\section{Interest rate}
Interest rate concern usually study the relation between zero rate $R(t,T)$, short rate $r_t$, zerocoupon price $B(t,T)$ and forward short rate $f(t,T)$ (see~\cite{RUDI}4.1)
\[
B(t,T)=e^{-R(t,T)(T-t)}
\hspace{1cm} \text{or} \hspace{1cm}
R(t,T) = -\frac{\text{ln}B(t,T)}{T-t}
\]
The short rate is defined as
\[
r_t = R(t,t) = -\lim_{\vartriangle t \rightarrow 0} \frac{\text{ln}B(t,t+\vartriangle t)}{\vartriangle t} = -\frac{\partial}{\partial T}\text{ln}B(t,t) 
\]
Let $B(t,T_1,T_2)$ and $R(t,T_1,T_2)$ be forward zero coupon and forward rate, then
\[
B(t,T_1,T_2) = \frac{B(t,T_2)}{B(t,T_1)}
\hspace{1cm} \text{and} \hspace{1cm}
B(t,T_1,T_2) = e^{-R(t,T_1,T_2)(T_2-T_1)}
\]
hence
\[
R(t,T_1,T_2) = - \frac{\text{ln}B(t,T_1,T_2)}{T_2-T_1}
\hspace{1cm} \text{or} \hspace{1cm}
R(t,T_1,T_2) = - \frac{\text{ln}B(t,T_2)-\text{ln}B(t,T_1)}{T_2-T_1}
\]
The forward short rate is defined as
\[
f(t,T)=R(t,T,T) = - \lim_{\vartriangle t \rightarrow 0}  \frac{\text{ln}B(t,T+\vartriangle t)-\text{ln}B(t,T)}{\vartriangle t}
=  -\frac{\partial}{\partial T}\text{ln}B(t,T) 
\]
The zerocoupon bond prices and the zero rate return can be deduced from forward short rate
\[
B(t,T) = e^{-\int_t^Tf(t,s)ds}
\hspace{1cm} \text{and} \hspace{1cm}
R(t,T) = \frac{1}{T-t}\int_t^T f(t,s)ds
\]
\paragraph{T-Forward price} for a claim $h_T\in \mathbb{L}^2(\mathcal{F}_T)$ is the the price \textit{agreed at time} $t\leq T$ that will be paid at time $T$ (see ~\cite{ELLIOT}ch.9)
\[
F(h_T,t,T) = \frac{\pi_t(h_T)}{B(t,T)}
\hspace{2cm}
\pi_t(h_T) \text{ is the risk-neutral price of } h_T \text{ at time } t
\]
In particular
\[
F(S_T,t,T) = \frac{S_t}{B(t,T)} \text{ for an asset,}
\hspace{2cm}
F(B(T_1,T_2),t,T_1) = \frac{B(t,T_2)}{B(t,T_1)}
\text{ , }
t\leq T_1 \leq T_2
\]
\paragraph{T-Forward measure} is the measure the equivalent measure using $B(t,T)$ as numeraire
\[
B(t,T)=S^0_t \mathbb{E}_{\mathbb{P}^*}[\frac{1}{S^0_T}|\mathcal{F}_t]
\hspace{2cm}
\frac{d\mathbb{Q}_T}{d\mathbb{P}^*} = \frac{1}{S^0 B(0,T)}
\]
For any $h_T\in \mathbb{L}^2(\mathcal{F}_T)$ contigent claim, its $T$-forward price is a $\mathbb{Q}_T$-martingale.

\section{Models}
Models are well explained in ~\cite{MAREK}
\subsection{Bachelier Model}
Instead of using the geometric brownian motion, Bachelier use the arithmetic brownian to model the stock price. See ~\cite{MAREK} ch3.3
\[
dS^1_t = \mu dt + \sigma_t d W_t   \hspace{1cm} dS^0_t =S^0_t r_t dt
\hspace{1cm} \rightarrow \hspace{2cm}
d\widetilde{S}^1_t = \frac{1}{S^0_t}( (\mu - rS^1_t) dt + \sigma_t d W_t )
\]
Girsanov Transform
\[
\theta_t = \frac{\mu - r_t S_t}{\sigma_t} 
\hspace{1cm} 
\frac{d\mathbb{P}^*}{d\mathbb{P}}|_{\mathcal{F}_t} = \text{exp}\{ -\int_0^t \theta_s dW_s - \frac{1}{2} \int_0^t \theta^2_s ds \}
\hspace{1cm} 
W^*_t = W_t + \int_0^t \theta_s ds
\]
Under martingale measure $\mathbb{P}^*$
\[
\begin{array}{l}
dS^1_t = rS^1_t dt + \sigma_t d W^*_t 
\hspace{3cm}
d\widetilde{S}^1_t =  \frac{1}{S^0_t} \sigma_t d W^*_t = \widetilde{\sigma}_t d W^*_t \\ \\
S^1_t = e^{rt}S^1_0 + \int^t_0 e^{r(t-u)} \sigma_t dW^*_u = e^{r(t-s)}S^1_s  + \int^t_s e^{r(t-u)} \sigma_t dW^*_u
\end{array}
\]
The Bachelier price formula
\[
C_t = e^{rt}\sigma\sqrt{T-t}\phi(d) + (S_t - K^{-r(T-t)})\Phi(d)
\hspace{1cm}
P_t=
\hspace{2cm}
d=\frac{ e^{r(T-t)}S_t -K  }{ e^{rT}\sigma\sqrt{T-t} }
\]
The Bachelier PDE

\[
\frac{\partial v}{\partial t} +
\frac{1}{2}\sigma^2_t \frac{\partial^2 v}{\partial s^2} +
r_t S_t \frac{\partial v}{\partial s} - 
r_t C_t =0
\hspace{2cm}
\text{ boundary : } v(s,T) = h(s)
\]
\subsection{Merton Model}
Instead of using the deterministic interest rate, Merton introduce a model for the zero-coupon proccess ~\cite{MAREK} ch3.1.7.
\[
\left\{ 
\begin{array}{rcl}
dS_t    &=& S_t( \mu_t dt + \sigma_t d W^{S}_t )  \\ 
dB(t,T) &=& B(t,T) ( a_t dt + b_t d W^{B}_t )
\end{array}\right. 
\hspace{1cm}
d<W^S,W^B> = \rho dt
\]
Where $S$ and $B(t,T)$ are stock and zero-coupon price proccess. The model's parameters $\mu_t, \sigma_t, a_t$ and $b_t$ are all supposed deterministic, functions in time.

\subsection{Black Model}
Black Model used for modelling the options on future. See ~\cite{MAREK} ch3.4, ~\cite{ELLIOT}.ch9
\subsection{Vasicek Model}
Vasicek's model is used for modelling the short rate, see ~\cite{ELLIOT}.ch9.6.
\[
dr_t = a (b-r_t) dt +\sigma  dW_t
\hspace{3cm}
r_0, a,b,\sigma >0
\]
The solution of the SDE is 
\[
r_t = e^{-at}( r_0 + b(e^{at} -1) + \sigma\int_0^t e^{au}dW_u   )
\]
This is a Urnstein-Uhlenbeck process. Where $b$ is the long term mean lever, $a$ is the speed of reversion, $\sigma$ is the volatility, $\frac{\sigma^2}{2a}$ is the long term variance. Vasicek's model allow the mean reversion proprety of the interest rate.

\subsection{Heath-Jarrow-Morton framework}
HJM model the forward short rate $f(t,T)$ where $t_0\leq t \leq T \leq T^*$.
\[
d_t f(t,T) = \mu(t,T)dt + \sigma(t,T)d\textbf{W}_t
\]

\subsection{Hull-White Model - interest rate}
Hull-White's Model is a generalization of Vasicek's model, see ~\cite{ELLIOT}.ch9.6. 
\[
dr_t = (\alpha_t - \beta_t r_t) dt +\sigma_t dW_t
\]
Where $\alpha_t$, $\beta_t$ and $\sigma_t$ are deterministic functions in time. The solution for the SDE is
\[
r_t = e^{-b_t} ( r_0 + \int_0^t e^{b_u}\alpha_u du + \int_0^t e^{b_u} \sigma_u dW_u   )
\hspace{2cm}
b_u = \int_0^u \beta_s ds 
\]
\subsection{Cox-Ingersoll-Ross model}
CIR model is used for modelling the short rate, see ~\cite{ELLIOT}.ch9.6. CIR better then Hull-White and Vasicek in the sens that the short rate in this model is always positive.
\[
dr_t = a (b-r_t) dt +\sigma \sqrt{r_t} dW_t
\hspace{3cm}
a>0,b>0
\]
CIR model has the same behavior of the Vasicek model. This process can be defined as a sum of squared Ornstein-Uhlenbeck process.

\subsection{Constant Elasticity of Variance Model}
CEV model is a local volatility model describing the asset price dynamic under the martingale measure $\mathbb{P}^*$as below
\[
dS_t = S_t ( r dt + \alpha S^{1-\beta}_t d W^*_t )   \hspace{2cm}  0< \alpha,0<\beta<1
\]
CEV model use the fact that the relative volatility $\alpha S^{1-\beta}_t$ depend to the actual level of stock price.

\subsection{Hull-White Model - volatility}
Hull-White model for volatility is a stochastic volatility model, where brownians are independent.
\[
\text{ under }\mathbb{P}, 
\hspace{1cm}
\left\{ 
\begin{array}{rcl}
dS_t    &=& S_t( \mu_t dt + \sigma_t d W^{S}_t )  \\ 
d\sigma_t  &=&  a(\sigma_t,t) dt + b(\sigma_t,t) d {W}^{\sigma}_t )
\end{array}\right. 
\hspace{2cm}
{W}^{\sigma}_t \perp {W}^{S}_t
\]
By Girsanov translating with asset price of risk and volatility price of risk
\[
\text{ under }\mathbb{P}^*, 
\hspace{1cm}
\left\{ 
\begin{array}{rcl}
dS_t    &=& S_t( r_t dt + \sigma_t d W^{*}_t )  \\ 
d\sigma_t  &=&  \widetilde{a}(\sigma_t,t) dt + b(\sigma_t,t) d \widetilde{W}_t )
\end{array}\right. 
\]
Note that under martingale measure, the two brownians are not necessarily independent.

\subsection{Heston Model}
Heston model is a stochastic volatility model. Further than giving the dynamic of underlying asset, Heston add the dynamic of the volatility (see ~\cite{MAREK}) under the historical probability $\mathbb{P}$
\[
\left\{ 
\begin{array}{rcl}
dS_t    &=& S_t( \mu_t dt + \sqrt{\nu_t} d \widetilde{W}^{1}_t )  \\ 
d\nu_t  &=& \hat{\kappa}( \hat{\nu} -\nu_t ) dt + \eta d \widetilde{W}^{2}_t )
\end{array}\right. 
\hspace{1cm}
d<\widetilde{W}^{1},\widetilde{W}^{2}> = \rho dt
\]
In this model, there are two market price of risk
\begin{itemize}
 \item $\theta = \frac{r-\mu}{\sqrt{\nu_t}}$ as usual in the Black Scholes model
 \item $\lambda=\alpha\sqrt{\nu_t}$ is the market price of volatility 
\end{itemize}
We can define a translated correlated Brownian by
\[
\left[
\begin{array}{c}
 \hat{W}^1  \\
 \hat{W}^2
\end{array}
\right]
=
\left[
\begin{array}{c}
 \widetilde{W}^1  \\
 \widetilde{W}^2
\end{array}
\right]
+
\left[
\begin{array}{c}
 \theta  \\
 \lambda
\end{array}
\right]
=
\left[
\begin{array}{c}
 \widetilde{W}^1  \\
 \widetilde{W}^2
\end{array}
\right]
+
\left[
\begin{array}{c}
 \frac{r-\mu}{\sqrt{\nu_t}}  \\
 \alpha\sqrt{\nu_t}
\end{array}
\right]
\text{ and }
\left[
\begin{array}{c}
 \frac{r-\mu}{\sqrt{\nu_t}}  \\
 \alpha\sqrt{\nu_t}
\end{array}
\right]
=
\left[
\begin{array}{cc}
 1    & 0  \\
 \rho & \sqrt{1-\rho^2}
\end{array}
\right]
\left[
\begin{array}{c}
 h^1\\
 h^2
\end{array}
\right]
\]
By inverting the correlation matrix, we find \textbf{h} and put into the Girsanov probability change, we find the risk-neutral probability $\mathbb{P}^*$ under what $\hat{\textbf{W}}_t$ is Brownian, and the EDS under $\mathbb{P}^*$ become
\[
\left\{ 
\begin{array}{rcl}
dS_t    &=& S_t( r dt + \sqrt{\nu_t} d \hat{W}^{1}_t )  \\ 
d\nu_t  &=& \kappa( \nu -\nu_t ) dt + \eta d \hat{W}^{2}_t )
\end{array}\right. 
\hspace{1cm}
d<\hat{W}^{1},\hat{W}^{2}> = \rho dt
\]

\subsection{SABR Model}
Stochastic-Alpha-Beta-Rho is a stochastic volatility model for the Forward price of an asset, or in particular for the Libor forward rate, interest rate derivatives. Such model has under the martingale measure the dynamic  
\[
\text{ under }\mathbb{P}^*, 
\hspace{1cm}
\left\{ 
\begin{array}{rcl}
dF_t       &=&     \alpha_t F^{\beta}_t d W^{*}_t )  \\ 
d\alpha_t  &=& \nu \alpha_t d \widetilde{W}_t )
\end{array}\right. 
\hspace{1cm}
d< W^{*},\widetilde{W}> = \rho dt
\]


\section{Useful mathematical formula}
- Ito multidim
- Girsanov multidim
- Faeymann Kac
- Martingale Representation theorem
- 
\section{Lagrange Multiplier}
To solve a optimization problem 
\[
\text{find maxima of } f(\textbf{x})  
\hspace{2cm}
\text{ subject to } g(\textbf{x}) = c
\]
This is equivalent to write down the Lagrangian and its equations
\[
\text{Lagrangian }\Lambda(\textbf{x},\lambda) = f(\textbf{x}) + \lambda(g(\textbf{x}) -c ) 
\hspace{2cm}
\text{ Solve } \triangledown_{(\textbf{x},\lambda)} \Lambda(\textbf{x},\lambda) = 0
\]

\section{Statistical Linear Gaussian Model}

\appendix
\section{Gaussian Vector}
\paragraph{Multivariate Normal Distribution}
A random vector $X\in \mathbb{R}^d$ has a multivariate normal distrubution it satisfied one of these conditions
\begin{enumerate}
 \item $\forall a \in \mathbb{R}^d, \hspace{3mm} a^{t}.X$ is a Gaussian variable
 \item There existe $\mu \in \mathbb{R}^d$ and $\Sigma \in \mathbb{R}^{d\times d}$ a \textbf{non-negative definite} matrix such that the characteristic function of $X$ is
 \[
  \varphi_{X}(u)     =  \text{exp}(i u^{t} .\mu  - \frac{1}{2} u^{t} .\Sigma. u) \hspace{10mm},\hspace{3mm} u\in\mathbb{R}^d
 \]
\end{enumerate}
The multivariate normal distribution is said to be \textbf{non-degenerate} iif the its covariance matrix $\Sigma$ is \textbf{positive definite}, and it has a density 
\[
f_{X}(x) = \frac{1}{ (2\pi)^{\frac{d}{2}} |\text{det}(\Sigma)|^{\frac{1}{2}} } \text{exp}\{ -\frac{1}{2} (x-\mu)^{t} \Sigma^{-1} (x-\mu) \}
\]
\paragraph{Cochran's theorem}
\[
\text{if }
\left\{
\begin{array}{l}
\textbf{X} = (X_1,X_2 ...,X_n) \sim \mathcal{N}(0, \sigma^2 \textbf{I}_n) \\
\textbf{E}_1\oplus\textbf{E}_2\oplus ... \textbf{E}_p\oplus = \mathbb{R}^n \text{ ortho. subspace} \\
\text{Dim}(\textbf{E}_i) = r_i   \\
\textbf{X}_{\textbf{E}_i} = P^{\textbf{E}_i}(\textbf{X}) \text{ ortho. project} 
\end{array}\right.
\text{ then }
\left\{
\begin{array}{l}
\textbf{X}_{\textbf{E}_i} \text{ are indept. and gaussian} \\
 \| \textbf{X}_{\textbf{E}_i} \|^{2} \text{ are indept. and } \sim \sigma^2\chi^2_{(r_i)}
\end{array}\right.
\]

\paragraph{Typical example}
\[
\left\{
\begin{array}{l}
\textbf{X} = (X_1,X_2 ...,X_n) \sim \mathcal{N}(0, \textbf{I}_n) \\ \\
\textbf{E}_1 = \text{Vect}(\frac{1}{\sqrt{n}} \mathbf{1}_n) \\ \\
\textbf{E}_2 = \textbf{E}_1^{\bot} \text{ , } \textbf{E}_1\oplus\textbf{E}_2= \mathbb{R}^n
\end{array}\right.
\hspace{2cm}
\left\{
\begin{array}{l}
 \sqrt{n} \overline{X}_n =  \textbf{X}_{\textbf{E}_1}\\ \\
 \sum^n_1 (X_i- \overline{X}_n) = \| \textbf{X}_{\textbf{E}_2} \|^{2} \\ \\
\overline{X}_n \text{ indept. } \sum^n_1 (X_i- \overline{X}_n) \\ \\
\sum^n_1 (X_i- \overline{X}_n) \sim \chi^2_{n-1}
\end{array}\right. 
\]



\bibliographystyle{siam}
\bibliography{QuantBib}
\end{document}
